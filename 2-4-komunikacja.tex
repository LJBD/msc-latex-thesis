\section{Komunikacja między elementami oprogramowania}
\label{sec:komunikacja}

\begin{itemize} 
    \item Część obliczeniowa wysyła:
    \begin{enumerate}
        \item wartości końcowe poziomów w zbiornikach (będące stanem docelowym w zadaniu czasooptymalnym oraz punktem linearyzacji modelu służącym do wyliczenia nastaw regulatora liniowo-kwadratowego),
        \item dla regulatora czasooptymalnego:
        \begin{enumerate}
            \item czasy przełączeń (założono postać sterowania typu ,,bang-bang''),
            \item wartość początkową sterowania czasooptymalnego,
            \item wartość ,,drugorzędną'' tego sterowania (założono, że część ,,niższa'' nie musi znać ograniczeń nałożonych na sterowanie),
            \item czas aplikacji sterowania czasooptymalnego (będący wartością wskaźnika jakości w tym zadaniu),
        \end{enumerate}
        \item dla regulatora liniowo-kwadratowego:
        \begin{enumerate}
            \item wektor K współczynników regulatora,
            \item sterowanie ustalone, od którego są liczone odchyłki.
        \end{enumerate}
    \end{enumerate}
    \item Część sterowania bezpośredniego wysyła:
    \begin{itemize}
        \item aktualne poziomy wody w zbiornikach,
        \item aktualną wartość sterowania.
    \end{itemize}
\end{itemize}