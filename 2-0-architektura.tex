\chapter{Architektura zaproponowanego rozwiązania}
\label{cha:arch}

W niniejszej pracy została zaproponowana dwuczęściowa struktura aplikacji realizującej zadania wyliczania i aplikowania sterowania czasooptymalnego, jak i minimalnego w sensie liniowo-kwadratowego wskaźnika jakości. Taka jej struktura jest powodowana faktyczną tendencją w automatyce ostatnich lat: aby skomplikowane zadania obliczeniowe zadawać nie tym elementom, które realizują bezpośrednie sterowanie, ale zlecać je innym urządzeniom o architekturze sprzętu odpowiedniejszej do radzenia sobie z takimi problemami.

\section{Podział zadań między elementami oprogramowania}
\label{sec:podzial-zadan}

W związku z założoną architekturą oprogramowania, opierającą się na dwóch częściach, kluczowym aspektem jest odpowiedni podział zadań między nimi oraz wybór parametrów wysyłanych przez obie strony.

Pierwsza z nich to część ,,wyższego poziomu'' - została tak określona ze względu na to, iż nie ma bezpośredniego wpływu na kontrolowany fizyczny obiekt. Jej zadania są przedstawione poniżej.

\begin{enumerate}
    \item Zadania części ,,wyższego poziomu'':
    \begin{enumerate}
        \item wyznaczanie sterowania czasooptymalnego dla zadanych wartości początkowych i końcowych,
        \item symulacja modelu nieliniowego, aby zweryfikować wyliczone sterowanie czasooptymalne,
        \item linearyzacja modelu w punkcie pracy, czyli w stanie docelowym zagadnienia czasooptymalnego,
        \item wyznaczanie sterowania optymalnego w sensie liniowo - kwadratowego wskaźnika jakości (dla modelu linearyzowanego w punkcie pracy).
    \end{enumerate}
    \item Wysyła ona:
    \begin{enumerate}
        \item wartości końcowe poziomów w zbiornikach (będące stanem docelowym w zadaniu czasooptymalnym oraz punktem linearyzacji modelu służącym do wyliczenia nastaw regulatora liniowo-kwadratowego),
        \item dla regulatora czasooptymalnego:
        \begin{enumerate}
            \item czasy przełączeń (założono postać sterowania typu ,,bang-bang''),
            \item wartość początkową sterowania czasooptymalnego,
            \item wartość ,,drugorzędną'' tego sterowania (założono, że część ,,niższa'' nie musi znać ograniczeń nałożonych na sterowanie),
            \item czas aplikacji sterowania czasooptymalnego (będący wartością wskaźnika jakości w tym zadaniu),
        \end{enumerate}
        \item dla regulatora liniowo-kwadratowego:
        \begin{enumerate}
            \item wektor K współczynników regulatora,
            \item sterowanie ustalone, od którego są liczone odchyłki.
        \end{enumerate}
    \end{enumerate}
\end{enumerate}

\section{Część obliczeniowa (,,wyższego poziomu'')}
\label{sec:czesc-wyzsza}

Aby zapewnić jak najlepszą realizację zadań postawionych przed częścią obliczeniową, należało wybrać odpowiednie oprogramowanie stanowiące podstawę tego poziomu aplikacji. Założono, że ta część powinna być napisana w języku programowania Python oraz używać 


%-------------------------------------------------
\subsection{Wybór pakietu do optymalizacji dynamicznej}
\label{sub:czesc-wyzsza-wybor}

Podstawowym elementem istotnym dla powodzenia takiego zadania był wybór pakietu realizującego operacje optymalizacji dynamicznej. Istnieje na rynku kilka płatnych bibliotek bądź pakietów oprogramowania, które są w stanie rozwiązywać tego typu problemy (np. MATLAB, Mathematica itp.), ale postanowiono poszukać darmowego odpowiednika, który posiadałby interfejs w języku programowania Python. Takie założenie zostało poczynione ze względu na to, iż biblioteką stanowiącą 


%-------------------------------------------------
\subsection{Opis pakietu Modelica}
\label{sub:czesc-wyzsza-modelica}


%-------------------------------------------------
\subsection{Opis systemu Tango Controls}
\label{sub:czesc-wyzsza-tango}


%-------------------------------------------------
\subsection{Architektura klasy urządzeń systemu Tango}
\label{sub:czesc-wyzsza-klasa}


\subsubsection{Interfejs dla systemu Tango}
%Lista komend, właściwości i atrybutów


\subsubsection{Problem dostępności interfejsu}

%Wątki vs procesy w Pythonie
%Propozycja uniwersalnego rozwiązania długich funkcji w DS Tango


%-------------------------------------------------
\subsection{Środowisko testowe części obliczeniowej}
\label{sub:czesc-wyzsza-docker}

\section{Część realizująca sterowanie (,,niższego poziomu'')}
\label{sec:czesc-nizsza}


%-------------------------------------------------
\subsection{Pakiet Simulink jako narzędzie realizujące sterowanie}
\label{sub:czesc-nizsza-matlab}

\cite{Trawinski2011}


%-------------------------------------------------
\subsection{Komunikacja z urządzeniem wykonawczym i czujnikami}
\label{sub:czesc-nizsza-komunikacja}


\section{Komunikacja między elementami oprogramowania}
\label{sec:komunikacja}


