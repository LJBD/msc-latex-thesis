\section{Część obliczeniowa (,,wyższego poziomu'')}
\label{sec:czesc-wyzsza}

Aby zapewnić jak najlepszą realizację zadań postawionych przed częścią obliczeniową, należało wybrać odpowiednie oprogramowanie używane przez ten poziom aplikacji. Założono, że ta część powinna być napisana w języku programowania Python oraz używać biblioteki Tango Controls jako swojej podstawy.


%-------------------------------------------------
\subsection{Opis systemu Tango Controls}
\label{sub:czesc-wyzsza-tango}

Tango Controls to zestaw narzędzi służący do zarządzania rozproszonymi systemami sterowania umożliwiający przyłączanie dowolnych urządzeń oraz fragmentów oprogramowania do jednej magistrali programowej. Jest on dystrybuowany na otwartej licencji (tzw. słabszej powszechnej licencji publicznej GNU - ang. \emph{Lesser GNU Public Licence} - w wersji 3) i utrzymywany przez konsorcjum placówek naukowych z całej Europy (więcj informacji w \cite{TangoWebsite}).

Początki tego oprogramowania sięgają roku 1999, kiedy to w Europejskim Ośrodku Synchrotronu Atomowego w Grenoble podjęto próbę implementacji nowego systemu pozwalającego na połączenie wszystkich urządzeń jedną magistralą programową i zaczęto pracę nad jej uogólnieniem, aby mogła przyjąć dowolny sprzęt, który tylko będzie miał napisany odpowiedni do niej sterownik. Osiągnięto ten cel dzięki użyciu odpowiedniej technologii komunikacyjnej (oryginalnie pakietu \texttt{omniORB} - implementacji architektury CORBA) oraz zastosowaniu własnego protokołu oraz ogólnej przestrzeni adresowej dla wszystkich urządzeń w systemie.

Przyłączanie nowego sprzętu do tego systemu opiera się na zasadzie opakowywania istniejącego kodu służącego do komunikacji z nim kodem realizującym operacje związane z systemem Tango Controls. W związku z tym, że jednym z podstawowych założeń tego systemu jest obiektowość, to opakowanie będzie miało postać ,,klasy urządzeń''.

Podstawowym pojęciem opisywanego systemu jest urządzenie, czyli instancja klasy urządzeń. Może to być:
\begin{itemize}
    \item logiczna abstrakcja istniejącego fizycznie sprzętu,
    \item logiczna abstrakcja grupy urządzeń,
    \item fragment oprogramowania (wtedy mówi się o czysto programowym urządzeniu).
\end{itemize}

Urządzenia systemu Tango są grupowane w serwery urządzeń - procesy w systemie operacyjnym, które przy swoim starcie pytają warstwy persystencji systemu (bazy danych typu MySQL) o to, jakie urządzenia mają uruchomić. Takie serwery dają również możliwość zarządzania urządzeniami w prosty sposób.
%TODO: klient/serwer - architektura tango

%TODO: urządzenie jako podstawowy koncept

Tabela \ref{tab:tango-implementations} podsumowuje wszystkie możliwe technologie, z których można korzystać, aby łączyć się z systemem zarządzanym przez bibliotekę Tango Controls.

\begin{table}[hpt]
    \centering
    \begin{tabular}{|c|c|c|}
        \hline 
        \textbf{Język} & \textbf{Typ implementacji} &\textbf{Co umożliwia} \\ 
        \hline 
        C++ & pełna & klient i serwer \\ 
        \hline 
        Java & pełna & klient i serwer \\ 
        \hline 
        Python & opakowanie do implementacji w C++ & klient i serwer \\ 
        \hline 
        C & opakowanie do implementacji w C++ & klient \\ 
        \hline 
        LabView & implementacja pomocnicza w języku C++ & klient i serwer \\ 
        \hline 
        MATLAB/Octave & tylko warstwa komunikacyjna napisana w języku C++ & klient \\ 
        \hline 
        IgorPro & tylko warstwa komunikacyjna napisana w języku C++ & klient \\ 
        \hline 
        JavaScript & tylko warstwa komunikacyjna z użyciem protokołu REST & klient \\ 
        \hline 
    \end{tabular}
    \caption{Podsumowanie języków programowania, w których istnieje możliwość połączenia z systemem Tango Controls. Źródło: \cite{TangoWebsite}.}
    \label{tab:tango-implementations}
\end{table}

%-------------------------------------------------
\subsection{Wybór pakietu do optymalizacji dynamicznej}
\label{sub:czesc-wyzsza-wybor}

Podstawowym elementem istotnym dla powodzenia takiego zadania był wybór pakietu realizującego operacje optymalizacji dynamicznej. Istnieje na rynku kilka płatnych bibliotek bądź pakietów oprogramowania, które są w stanie rozwiązywać tego typu problemy (np. MATLAB, Mathematica itp.), ale postanowiono poszukać darmowego odpowiednika, który posiadałby interfejs w języku programowania Python (w związku z ogólnymi założeniami co do struktury tej części aplikacji).


\subsubsection{Opis pakietu ACADO}


\subsubsection{Opis pakietu Modelica}



%-------------------------------------------------
\subsection{Architektura klasy urządzeń systemu Tango}
\label{sub:czesc-wyzsza-klasa}


\subsubsection{Interfejs dla systemu Tango}
%Lista komend, właściwości i atrybutów


\subsubsection{Problem dostępności interfejsu}

%Wątki vs procesy w Pythonie
%Propozycja uniwersalnego rozwiązania długich funkcji w DS Tango


%-------------------------------------------------
\subsection{Środowisko testowe części obliczeniowej}
\label{sub:czesc-wyzsza-docker}