\section{Część obliczeniowa (,,wyższego poziomu'')}
\label{sec:czesc-wyzsza}

Aby zapewnić jak najlepszą realizację zadań postawionych przed częścią obliczeniową, należało wybrać odpowiednie oprogramowanie używane przez ten poziom aplikacji. Przyjęto dwa podstawowe założenia co do niego:
\begin{itemize}
    \item powinna być napisana w języku programowania Python,
    \item powinna używać biblioteki Tango Controls jako swojej podstawy.
\end{itemize}

Pierwsze założenie zostało przyjęte ze względu na niewątpliwe zalety tego języka. Jest on mocno zorientowany na programowanie obiektowe, posiada przejrzystą, nieskomplikowaną składnię oraz, jako język interpretowany, umożliwia szybkie prototypowanie i korzystanie z interaktywnego wiersza poleceń.

Dodatkowo jest to język dostępny w całości na otwartej licencji, a społeczność skupiona wokół niego dostarcza wielu bibliotek ogólnego i szczególnego zastosowania, które również często są otwartym oprogramowaniem. Podjęto w związku z tym próbę znalezienia wśród tych bibliotek takich, które byłyby pomocne w rozwiązywaniu zagadnień optymalizacji dynamicznej - więcej na ten temat w sekcji \ref{sub:czesc-wyzsza-wybor}.

Drugie założenie zostanie wyjaśnione w kolejnej sekcji.

%-------------------------------------------------
\subsection{Opis systemu Tango Controls}
\label{sub:czesc-wyzsza-tango}

Tango Controls to zestaw narzędzi służący do zarządzania rozproszonymi systemami sterowania umożliwiający przyłączanie dowolnych urządzeń oraz fragmentów oprogramowania do jednej magistrali programowej. Jest on również dystrybuowany na otwartej licencji (tzw. słabszej powszechnej licencji publicznej GNU - ang. \emph{Lesser GNU Public Licence} - w wersji 3) i utrzymywany przez konsorcjum placówek naukowych z całej Europy (więcj informacji w \cite{TangoWebsite}).


\subsubsection{Ogólne założenia systemu}
Początki tego oprogramowania sięgają roku 1999, kiedy to w Europejskim Ośrodku Synchrotronu Atomowego w Grenoble podjęto próbę implementacji nowego systemu pozwalającego na połączenie wszystkich urządzeń jedną magistralą programową i zaczęto pracę nad jej uogólnieniem, aby mogła przyjąć dowolny sprzęt, który tylko będzie miał napisany odpowiedni do niej sterownik. Osiągnięto ten cel dzięki użyciu odpowiedniej technologii komunikacyjnej (oryginalnie implementacji architektury CORBA, od wersji 7 również biblioteki ZeroMQ) oraz zastosowaniu własnego protokołu oraz ogólnej przestrzeni adresowej dla wszystkich urządzeń w systemie. Zastosowano tutaj również podejście ,,klient - serwer'', a więc istnieje wyraźne rozgranicznie między interfejsem dla klienta (aplikacji pobierającej dane z systemu) oraz serwera (aplikacji dostarczającej dane). Dostępne są 3 tryby komunikacji między klientami a serwerami:
\begin{itemize}
    \item synchroniczny,
    \item asynchroniczny,
    \item zdarzeniowy.
\end{itemize}

Przyłączanie nowego sprzętu do tego systemu opiera się na zasadzie opakowywania istniejącego kodu służącego do komunikacji z nim kodem realizującym operacje związane z systemem Tango Controls. W związku z tym, że jednym z podstawowych założeń tego systemu jest obiektowość, to opakowanie będzie miało postać \emph{klasy urządzeń}.
Dzięki temu przejmuje on wszystkie zalety (oraz pułapki) programowania obiektowego: możliwość dziedziczenia między klasami, co ułatwia utrzymywanie hierarchicznej struktury pisanego oprogramowania, jak i możliwość współdzielenia bardziej ogólnych klas między różnymi systemami, które korzystają z Tango Controls. W szczególności wszystkie klasy muszą dziedziczyć po bazowej klasie - \emph{Device}.

\subsubsection{Urządzenia i ich serwery}
Podstawowym pojęciem opisywanego systemu jest \emph{urządzenie}, które jest obiektem w sensie programistycznym (instancją klasy danego typu urządzeń). Może to być:
\begin{itemize}
    \item logiczna abstrakcja istniejącego fizycznie sprzętu, niezależnie od tego, czy jest to jeden prosty czujnik, czy bardziej skomplikowane urządzenie z własnym HMI (ang. \emph{Human-Machine Interface}: interfejs człowiek-maszyna),
    \item logiczna abstrakcja grupy urządzeń, również niezależnie od ich stopnia skomplikowania,
    \item fragment oprogramowania - mówi się wtedy o czysto programowym urządzeniu.
\end{itemize}

Definicja przestrzeni adresowej systemu zakłada, że każde urządzenie w systemie ma nazwę składającą się z 3 członów przedzielonych prawymi ukośnikami (,,/''):
\begin{itemize}
    \item domeny,
    \item rodziny,
    \item członka.
\end{itemize}
Przykładowa nazwa wygląda następująco: \texttt{sys/tg\_test/1}.

Dodatkowo podając nazwę urządzenia, można również podać adres bazowy systemu, czyli adres i port, na którym można nawiązać komunikację TCP/IP z urządzeniem zarządzającym dostępem do warstwy persystencji systemu - bazy danych MySQL. Tam przechowywane są informacje na temat urządzeń kiedykolwiek uruchomionych w systemie. Taki adres bazowy systemu określa się jako zmienną środowiskową w systemie operacyjnym o nazwie \texttt{TANGO\_HOST}. Określenie takiej pary adresu i portu jednoznacznie definiuje, z którym systemem zarządzanym przez Tango Controls klient się chce połączyć. Uzupełniona o nią (i o nazwę protokołu) przykładowa nazwa urządzenia wygląda następująco: \texttt{tango://tango-host.org:10000/sys/tg\_test/1}.

Urządzenia systemu Tango są grupowane w \emph{serwery urządzeń} - procesy w systemie operacyjnym, które przy swoim starcie pytają warstwy persystencji o to, jakie urządzenia mają uruchomić. Każdy serwer urządzeń ma określony zestaw klas urządzeń, których instancjami może zarządzać: uruchamiać je, edytować, wyłączać, kasować itp. Umożliwia on również włączenie odpytywania konkretnego elementu interfejsu urządzenia z okresem zadanym w milisekundach. Proponuje się rozróżnienie na klasę urządzeń i typ serwera urządzeń, aby uniknąć nieścisłości. Każdy serwer jest również obiektem klasy urządzeń \emph{DServer}, a jego typ określa to, jakimi klasami urządzeń dysponuje.

Na interfejs urządzenia składają się cztery typy pól w klasie tego urządzenia:
\begin{enumerate}
    \item Atrybuty (ang. \emph{attributes}) - reprezentują wielkości fizyczne, zmieniające się w czasie działania urządzenia. Muszą mieć zdefiniowany typ (jeden z podstawowych typów danych numerycznych lub ciągu znaków) oraz wymiar (Tango Controls obsługuje atrybuty skalarne, wektorowe i macierzowe). Każde urządzenie musi mieć przynajmniej 2 atrybuty: stan i status (testowy opis tego, co się dzieje z urządzeniem). Przykładowy atrybut urządzenia będącego odwzorowaniem silnika krokowego w systemie Tango to pozycja (zmiennoprzecinkowa wartość skalarna) lub stan wyłączników krańcowych (dwuelementowy wektor wartości logicznych).
    
    \item Strumienie (ang. \emph{pipe} - pozwalają na przesyłanie dowolnego typu surowych danych do urządzenia lub na jego zewnątrz. Dla każdej ich paczki trzeba jednak określić ten typ przed transmisją. Mogą być jedno- lub dwukierunkowe. Jest to nowy typ pola w klasach urządzeń, obecny od wersji 9.
    
    \item Komendy (ang. \emph{commands})- będące operacjami, których wykonanie urządzenie umożliwia. Mogą być wywoływane z maksymalnie jednym argumentem wejściowym i również tylko jeden argument wyjściowy mogą zwracać. Przykładem komendy urządzenia będącego logiczną abstrakcją silnika krokowego jest operacja bazowania - znalezienie określonej pozycji sprzężonego enkodera i wyzerowanie rejestru położenia.
    
    \item Właściwości (ang. \emph{properties}) - zawierające statyczną konfigurację urządzenia. Jako jedyne są przechowywane w bazie danych, skąd serwer urządzeń przy starcie pobiera ich wartości.
\end{enumerate}

\begin{figure}[tph]
    \centering
    \includegraphics[width=\textwidth]{Grafika/tango_model}
    \caption{Uproszczony model obiektów w systemie Tango Controls. Źródło: \cite{TangoDocs}.}
    \label{fig:tango-model}
\end{figure}

Klasa urządzeń może zawierać definicję maszyny stanów, która opisuje warunki przejść między dowolnymi z 14 dostępnych stanów. Są w tym zbiorze ogólne stany opisujące poprawne działanie urządzenia (INIT, ON, OFF, STANDBY, RUNNING, MOVING), błędy w funkcjonowaniu (ALARM, FAULT, DISABLE), takie nadające się tylko dla określonych typów sprzętu (OPEN, CLOSE, INSERT, EXTRACT) oraz stan nieustalony (UNKNOWN). Maszyna stanów może również ograniczać dostęp do innych elementów interfejsu urządzenia (atrybutów i komend) na podstawie tego, w jakim stanie ono się znajduje.

Wszystkie opisane wyżej pojęcia oraz relacje między nimi zostały przedstawione na diagramie \ref{fig:tango-model}. Znajduje się tam klasa urządzeń (ang. \emph{Device Class}, \emph{TANGO Class}), która definiuje elementy interfejsu urządzeń będących jej instancjami (atrybuty, komendy i strumienie).
Na schemacie zostały dodatkowo oznaczone zdarzenia (ang. \emph{Events}), których każdy atrybut może wysyłać 3 typy: okresowe, archiwizacji i zmiany. Dodatkowo zaznaczono, że urządzenie musi posiadać atrybuty stanu i statusu.
Urządzenie jest obiektem takiej klasy i należy do serwera urządzeń (ang. \emph{Device Server}). Każde dodatkowo ma konkretne wartości właściwości przechowywane w bazie danych systemu Tango.

\subsubsection{Interfejs programistyczny systemu}
Specyfikacja Tango Controls definiuje API (ang. \emph{Application Programming Interface}: interfejs programowania aplikacji), do którego jest dostęp w różnych językach programowania na wielu systemach operacyjnych (m.in.: różne dystrybucje systemu Linux, Windows, Mac OS oraz Solaris). Jest on dokładnie opisany w dokumentacji dostępnej w \cite{TangoDocs}.

Tabela \ref{tab:tango-implementations} podsumowuje wszystkie możliwe technologie, z których można korzystać, aby łączyć się z systemem zarządzanym przez Tango Controls.

Oprócz tego istnieje też bogaty zestaw narzędzi służących do zarządzania systemem, nadzorowania go, tworzenia interfejsów graficznych oraz realizowania innych usług wymaganych w rozproszonym systemie sterowania (np. archiwizacji, zbierania logów czy automatycznej konfiguracji).

\begin{table}[hpt]
    \centering
    \begin{tabular}{|c|c|c|}
        \hline 
        \textbf{Język} & \textbf{Typ implementacji} &\textbf{Co umożliwia} \\ 
        \hline 
        C++ & pełna & klient i serwer \\ 
        \hline 
        Java & pełna & klient i serwer \\ 
        \hline 
        Python & opakowanie do implementacji w C++ & klient i serwer \\ 
        \hline 
        C & opakowanie do implementacji w C++ & klient \\ 
        \hline 
        LabView & implementacja pomocnicza w języku C++ & klient i serwer \\ 
        \hline 
        MATLAB/Octave & tylko warstwa komunikacyjna w języku C++ & klient \\ 
        \hline 
        IgorPro & tylko warstwa komunikacyjna w języku C++ & klient \\ 
        \hline 
        Panorama & wtyczka do aplikacji & klient \\ 
        \hline 
        JavaScript & tylko warstwa komunikacyjna z użyciem protokołu REST & klient \\ 
        \hline 
    \end{tabular}
    \caption{Podsumowanie języków programowania, w których istnieje możliwość połączenia z systemem Tango Controls. Źródło: \cite{TangoWebsite}.}
    \label{tab:tango-implementations}
\end{table}

W związku ze wszystkimi opisanymi cechami uznano Tango Controls za odpowiednią bibliotekę do oparcia na niej części obliczeniowej aplikacji stanowiącej podstawę niniejszej pracy.


%-------------------------------------------------
\subsection{Architektura klasy urządzeń systemu Tango}
\label{sub:czesc-wyzsza-klasa}

Architektura systemu Tango Controls wymusiła pewne decyzje co do struktury tej części aplikacji. Zdecydowano, że będzie ona napisana jako klasa urządzeń, która określa interfejs dla wszystkich operacji wymaganych do spełnienia zadań postawionych przed tym poziomem aplikacji w podrozdziale \ref{sec:podzial-zadan}. Nie zdecydowano się na podział zadań na mniejsze klasy urządzeń komunikujące się ze sobą. Przyjęto, że prostsza struktura z jedną klasą ułatwi użycie napisanego kodu w przyszłości, np. w formie części zajęć laboratoryjnych.

Dokumentacja kodu klasy urządzeń jest wpisana w pliki źródłowe. Można również zbudować ją osobno, na przykład w środowisku testowym dla wyższego poziomu aplikacji, którego opis znajduje się w sekcji \ref{sub:czesc-wyzsza-docker}.

Dodatkowym elementem wyższego poziomu aplikacji jest serwer TCP, który zarządza komunikacją z aplikacją sterowania bezpośredniego. Został on zaimplementowany jako osobny moduł uruchamiany z poziomu urządzenia systemu Tango.

\subsubsection{Interfejs dla systemu Tango}

Określając interfejs dla systemu Tango, przyjęto założenie, że wszystkie parametry statyczne modelu matematycznego powinny być również statyczną konfiguracją urządzenia, a więc należy opisać je jako jego właściwości. Głównym powodem przyjęcia takiej reguły jest fakt, iż w czasie działania - realizacji sterowania optymalnego - te wielkości nie będą się zmieniać.
Parametry dynamiczne, czyli takie, które zmieniają się między uruchomieniami algorytmów optymalizacyjnych, powinny zostać opisane jako atrybuty urządzenia.
Wszystkie dostępne akcje na modelu matematycznym zostały zaimplementowane jako komendy. Dodano tam również wszystkie akcje opisane w podrozdziale \ref{sec:komunikacja}.

Poniższa lista zawiera wszystkie istotne dla użytkownika elementy interfejsu systemu Tango urządzenia realizującego zadania wyższego poziomu aplikacji. Każdy z nich jest pokrótce opisany.

\begin{enumerate}
    \item Właściwości:
    \begin{enumerate}
        \item \emph{IPOPTTolerance} - tolerancja błędów oprogramowania do rozwiązywania zagadnień algebry liniowej (więcej o wpływie na działanie optymalizacji w sekcji \ref{sub:opt-dokladnosc}).
        \item \emph{ModelFile} - nazwa pliku z modelem dla oprogramowania do optymalizacji (więcej w sekcji \ref{sub:czesc-wyzsza-wybor}).
        \item \emph{MaxControl} - górne ograniczenie sterowania.
        \item \emph{Tank*Outflow} - opory wypływu ze zbiorników $C_{i}$. Dane jako 3 osobne właściwości - w miejscu * jest liczba 1, 2 lub 3.
        \item \emph{Tank*FlowCoeff} - współczynniki wypływu ze zbiorników $\alpha_{i}$. Dane j.w.
        \item \emph{SimulationFinalTime} - czas symulacji używanej w celu inicjalizacji algorytmu optymalizacji (więcej o tym procesie w sekcji \ref{sub:opt-init}.
        \item \emph{TCPServerEnabled} - wartość logiczna zawierająca informację o tym, czy urządzenie ma uruchomić serwer TCP w celu komunikowania się z aplikacją niższego poziomu.
        \item \emph{TCPServerAddress} - para adres i port (przedzielone dwukropkiem). Pod tym adresem serwer TCP będzie nasłuchiwał przychodzących połączeń.
    \end{enumerate}
    \item Atrybuty (nie umożliwiają zapisu, chyba że zaznaczono inaczej):
    \begin{enumerate}
        \item \emph{H*Current} - zawierają aktualne poziomy wody w zbiornikach odebrane od aplikacji niższego poziomu. Dane jako 3 osobne atrybuty - w miejscu * jest liczba 1, 2 lub 3.
        \item \emph{H*Initial} - zawierają początkowe wartości poziomów dla optymalizacji dynamicznej. Dane j.w., możliwy zapis.
        \item \emph{H*Final} - zawierają końcowe wartości poziomów dla optymalizacji dynamicznej. Dane j.w., możliwy zapis.
        \item \emph{H*Simulated} - zawierają wektory przebiegu poziomów w zbiornikach uzyskane w symulacji. Dane j.w.
        \item \emph{OptimalH*} - zawierają wektory przebiegu poziomów w zbiornikach uzyskane w czasie optymalizacji (wyznaczania sterowania czasooptymalnego). Dane j.w.
        \item \emph{OptimalTime} - liczba będąca wartością wskaźnika jakości uzyskaną w procesie wyznaczania sterowania czasooptymalnego.
        \item \emph{SimulationTime} - wektor wartości czasu uzyskany w czasie symulacji potrzebny w celu rysowania przebiegów symulacyjnych \emph{H*Simulated}.
        \item \emph{OptimalControl} - wektor zawierający sterowanie czasooptymalne.
        \item \emph{ControlCurrent} - zawiera aktualną wartość sterowania odebraną od aplikacji niższego poziomu.
        \item \emph{SwitchTimes} - wektor zawierający czasy przełączeń po normalizacji sterowania czasooptymalnego.
        \item \emph{Q} - macierz 3 na 3 zawierająca wagi poziomów do zagadnienia liniowo-kwadratowego. Możliwy zapis.
        \item \emph{R} - wartość wagi sterowania do zagadnienia liniowo-kwadratowego. Możliwy zapis.
        \item \emph{K} - wektor współczynników regulatora liniowo-kwadratowego.
        \item \emph{VerificationError} - wartość błędu weryfikacj (więcej o jego wyznaczaniu w podrozdziale \ref{sec:sym-wer}).
        \item \emph{FEMElements} - liczba elementów metody elementów skończonych, używanej przez algorytm optymalizacyjny. Możliwy zapis.
        \item Dostępne stany:
        \begin{enumerate}
            \item OFF - model początkowy jest niewczytany.
            \item STANDBY - model początkowy jest wczytany, ale nie trwa żadna długa operacja.
            \item ON - trwa symulacja.
            \item RUNNING - trwa optymalizacja.
            \item ALARM - długa operacja się nie udała. W przypadku optymalizacji oznacza to, iż nie udało się znaleźć sterowania optymalnego. W przypadku weryfikacji, że zwróciła ona błąd (więcej na ten temat w sekcji \ref{sub:sym-wer-jmodelica}).
        \end{enumerate}
    \end{enumerate}
    \item Komendy:
    \begin{enumerate}
        \item \emph{LoadInitialModel} - wczytuje model początkowy służący do inicjalizacji symulacji i optymalizacji.
        \item \emph{ResetModel} - resetuje wczytany model początkowy.
        \item \emph{GetEquilibriumFromControl} - przyjmuje wartość sterowania i na jej podstawie wylicza punkt równowagi, a następnie ustawia odpowiednie wartości poziomów końcowych.
        \item \emph{RunSimulation} - uruchamia symulację, a po jej zakończeniu ustawia wartości zmiennych symulacyjnych.
        \item \emph{RunVerification} - uruchamia symulację, a po jej zakończeniu ustawia wartości zmiennych symulacyjnych oraz wylicza błąd weryfikacji.
        \item \emph{Optimise} - uruchamia procedurę optymalizacji, a następnie ustawia wartości poziomów optymalnych, sterowania optymalnego i osiągniętego czasu. Przyjmuje jeden argument - wartość logiczną, która określa, czy jako wartości początkowych powinna użyć aktualnych poziomów wody w zbiornikach. Jeśli tak się dzieje, to natychmiast po zakończeniu optymalizacji uruchamia komendę \emph{SendControl}.
        \item \emph{NormaliseOptimalControl} - przeprowadza operację normalizacji sterowania czasooptymalnego (więcej na ten temat w sekcji \ref{sub:opt-dokladnosc}) i ustawia wartości czasów przełączeń.
        \item \emph{SendControl} - sprawdza, czy są gotowe wartości do wysłania do aplikacji niższego poziomu, czyli czasy przełączeń i wartości wektora $K$ (jeśli nie są gotowe, uruchamia komendę \emph{GetLQR}), a następnie wysyła je do serwera TCP.
        \item \emph{GetDataFromDirectControl} - odbiera od serwera TCP dane od aplikacji sterowania bezpośredniego. Jest odpytywana standardowo co 50 milisekund. Ustawia aktualne poziomy i sterowanie.
        \item \emph{GetLQR} - uruchamia procedurę linearyzacji i wyliczania parametrów regulatora liniowo-kwadratowego. Ustawia wartości wektora K.
        \item \emph{GetEigenvaluesFromClosedLoopWithLQR} - zwraca w formie tekstowej wartości własne macierzy $A - BR^{-1}B^{T}K$ opisującej układ zamknięty sterowany regulatorem liniowo-kwadratowym.
    \end{enumerate}
\end{enumerate}

%-------------------------------------------------
\subsection{Wybór pakietu do optymalizacji dynamicznej}
\label{sub:czesc-wyzsza-wybor}

Podstawowym elementem istotnym dla powodzenia takiego zadania był wybór pakietu realizującego operacje optymalizacji dynamicznej. Istnieje na rynku kilka płatnych bibliotek bądź pakietów oprogramowania, które są w stanie rozwiązywać tego typu problemy (np. MATLAB, Mathematica itp.), ale postanowiono poszukać darmowego odpowiednika, który posiadałby interfejs w języku programowania Python (w związku z ogólnymi założeniami co do struktury tej części aplikacji).


\subsubsection{Opis pakietu ACADO}

Pierwszym z pakietów optymalizacyjnych, który był rozważany, jest ACADO (ang. \emph{Automatic Control and Dynamic
Optimization}). Jest to zestaw algorytmów napisanych w języku C++ służących do różnych zastosowań związanych ze sterowaniem optymalnym, dostępny na otwartej licencji (za: \cite{Houska2011}). W jego skład wchodzą między innymi metody:
\begin{itemize}
    \item bezpośrednie optymalizacji dynamicznej (w tym wielokryterialnej):
        \begin{itemize}
            \item programowanie nieliniowe (w tym programowanie kwadratowe: ang. \emph{QP - Quadratic Programming}),
            \item bezpośrednia metoda strzałów wielokrotnych;
        \end{itemize}
    \item sterowania predykcyjnego (MPC, ang. \emph{Model Predictive Control}),
    \item estymacji parametrów i stanu.
\end{itemize}

Dodatkowo opisywane oprogramowanie umożliwia generowanie kodu w języku C rozwiązującego konkretny problem, który może być użyty w mikrokontrolerze. Oprócz tego istnieje również możliwość użycia go w programie MATLAB, jako dodatkowego pakietu.

Twórcom tej biblioteki przyświecały cztery główne założenia:
\begin{enumerate}
    \item Oprogramowanie musi być dostępne na otwartej licencji.
    \item Składnia musi być jak najbardziej intuicyjna, aby nie odrzucać potencjalnych użytkowników (którzy mogą nie mieć dużego doświadczenia z językiem C++) ilością kodu szablonowego (ang. \emph{boilerplate code}). Powinna umożliwiać prostą definicję problemów sterowania optymalnego. Udało się to osiągnąć, stosując różne bardziej zaawansowane elementy składni języka programowania, w szczególności przeładowywania operatorów, aby składnia była jak najbardziej zbliżona do matematycznego zapisu.
    \item Kod powinien być napisany w taki sposób, aby ułatwić rozszerzanie go o własne algorytmy oraz używanie metod z tej biblioteki w innych, większych projektach. To założenie jest spełnione przez odpowiednie zaplanowanie struktury oprogramowania oraz przestrzeganie paradygmatu programowania obiektowego.
    \item Brak zewnętrznych zależności. Użytkownik może zastosować ACADO, jeśli tylko posiada kompilator języka C++. Dodatkowe biblioteki nie są wymagane, ale mogą być użyte przy rysowaniu wykresów oraz specjalistycznych obliczeniach algebraicznych.
\end{enumerate}

Aby zilustrować prostotę składni, poniżej załączono definicję modelu matematycznego opisywanego w niniejszej pracy zestawu zbiorników. Znajduje się tu definicja stanu, sterowania i parametrów (linie 4 - 6), samego równania różniczkowego (linie 8 - 15) oraz problemu optymalizacji wraz z ograniczeniami (linie 20 - 38).

\lstinputlisting[language=C++, firstline=21, lastline=58]{../masthe-acado-tanks-optimisation/tanks-optimisation.cpp}

Mimo iż pakiet ACADO jest napisany w języku C++, a nie w języku Python, została przeprowadzona próba użycia go do rozwiązania problemu znajdowania sterowania czasooptymalnego w opisywanym w niniejszej pracy układzie zbiorników. Wykorzystano tutaj metodę bezpośrednią opartą na programowaniu nieliniowym. Niestety, nie zakończyła się ona powodzeniem: bez względu na zmiany opcji algorytmu (w tym tolerancji rozwiązania) oraz na zastosowane techniki inicjalizacji nie udało się przeprowadzić ani jednej poprawnej optymalizacji. W związku z tym zdecydowano się zrezygnować z dalszych prób użycia pakietu ACADO i poszukać oprogramowania bardziej odpowiedniego do rozpatrywanego problemu.

\subsubsection{Opis pakietu Modelica i JModelica.org}

Dalsze poszukiwania oprogramowania do optymalizacji dynamicznej przyniosły odkrycie pakietu Modelica. Jest to obiektowy język opisu dynamiki układów fizycznych oparty na definicji równań i matematycznych zależności między nimi (za: \cite{ModelicaWebsite}). Pozwala on na definicję równań różniczkowych i algebraicznych, zarówno ciągłych, jak i dyskretnych.
Celem przyświecającym jego twórcom jest dostarczenie narzędzia służącego do prostego przedstawiania w formie tekstowej i graficznej komponentów układów fizycznych:
\begin{itemize}
    \item mechanicznych (jedno-, dwu- i trójwymiarowych),
    \item elektrycznych,
    \item termodynamicznych,
    \item hydraulicznych,
    \item pneumatycznych.
\end{itemize}

Opisywany układ może zawierać dowolną kombinację komponentów każdego z tych rodzajów. Każdy z nich ma odpowiadającą sobie część biblioteki standardowej języka Modelica, która oprócz tego zawiera również komponenty przydatne w opisie złożonych systemów (m.in definicje jednostek układu SI, maszyn stanów, zaawansowanych funkcji matematycznych czy stałych fizycznych). Zgodnie z paradygmatem obiektowości, wszystkie komponenty można rozszerzać, aby tworzyć własne ich wariacje.
Całe oprogramowanie jest dostępne na własnej, otwartej licencji, która również może być wykorzystywana do licencjonowania modeli napisanych w tym języku.

Poniżej podano definicję modelu opisywanego w niniejszej pracy układu zbiorników w języku Modelica w formie tekstowej, aby zaprezentować jego składnię. Przygotowano paczkę \texttt{TanksPkg} rozszerzającą bibliotekę standardową o następujące komponenty:
\begin{itemize}
    \item model częściowy zawierający definicje parametrów i równań różniczkowych opisujących układ (na podstawie wzoru \ref{eq:model}), ale bez wejścia sterowania $u$ (linie 3 - 33),
    \item model podstawowy rozszerzający powyższy o definicję wejścia sterowania (linie 35 - 37),
    \item model inicjalizacyjny służący do wyznaczania punktów równowagi układu (linie 39 - 45).
\end{itemize}

Każdy z nich zawiera definicję parametrów i zmiennych określającą ich typy (i jednostki, jeśli to możliwe) oraz równania (lub równania początkowe w przypadku modelu inicjalizacyjnego). Wyjątkiem jest tutaj tylko model podstawowy, który tylko rozszerza model częściowy i nie zawiera żadnych dodatkowych definicji poza redeklaracją typu parametru \texttt{u}.

\lstinputlisting[language=modelica, firstline=2, lastline=45]
{../masthe-tango-ds-tanks-regulation/optimica_model/opt_3_tanks.mop}

Modelica jest wykorzystywana jako podstawa opisu modeli w różnego rodzaju komercyjnym oprogramowaniu do symulacji (m.in. CATIA, Dymola czy MapleSim), ale są dostępne również jego bezpłatne odpowiedniki: OpenModelica i JModelica.org. Drugi z nich zasługuje na szczególną uwagę ze względu na dwa względy:
\begin{itemize}
    \item rozszerzenie biblioteki Modelica o opis problemów optymalizacji,
    \item dostępność API w języku Python do modeli napisanych w języku Modelica.
\end{itemize} 
Ze względu na te fakty podjęta została próba wykorzystania go w wyższym poziomie aplikacji.

JModelica.org jest platformą do symulacji, optymalizacji i analizy układów opisanych w języku Modelica. Jak wspomniano wcześniej, dostarcza również możliwości opisu problemów optymalizacji dynamicznej i metody do ich rozwiązywania zebrane we wspólnym pakiecie Optimica (dokładniejszy opis w \cite{JModelicaUserGuide}). Jest dostępna na otwartej licencji (GPL w wersji 3) i utrzymywana przez firmę Modelon AB, choć sam pomysł stworzenia takiej platformy powstał na Uniwersytecie w Lund (za: \cite{JModelicaWebsite}).

\begin{figure}[htp]
    \centering
    \includegraphics[width=\textwidth]{Grafika/jmodelica_architecture}
    \caption{Schemat architektury platformy JModelica.org. Źródło: \cite{JModelicaWebsite}.}
    \label{fig:jmodelicaarchitecture}
\end{figure}

Architektura platformy JModelica.org składa się następujących komponentów (pokazanych schematycznie na rys. \ref{fig:jmodelicaarchitecture}):
\begin{itemize}
    \item kompilatora, który sprawdza poprawność modeli opisanych w językach Modelica i Optimica, a następnie generuje kod w języku C zawierający równania układów i pliki w formacie XML z metadanymi,
    \item biblioteki wykonywalnej JModelica.org napisanej w języku C, która pozwala skompilować model opisany w języku C i dostarczyć go do zewnętrznego oprogramowania do wyliczania pochodnych w formie przez nie zrozumiałej,
    \item biblioteki w językach C++, Java i Python zawierającej implementacje metod optymalizacji dynamicznej (używana jest tu metoda kolokacji bezpośredniej opisana w \cite{ake+10cace}),
    \item API do języka Python dającego użytkownikowi dostęp do wszystkich pozostałych komponentów; w jego skład wchodzą 3 pakiety:
    \begin{itemize}
        \item \emph{PyModelica} dostarcza interfejs do kompilatorów języków Modelica i Optimica,
        \item \emph{PyFMI} umożliwia manipulację parametrami skompilowanych modeli i ich symulację,
        \item \emph{PyJMI} służy jako interfejs do algorytmów optymalizacji;
    \end{itemize}
    \item wtyczki do środowiska programistycznego Eclipse.
\end{itemize}

Poniżej pokazano definicję problemu optymalizacji rozważanego w niniejszej pracy napisaną w języku Optimica.
Jest ona rozszerzeniem modelu podstawowego opisanego wyżej i zawiera dodatkowe definicje wskaźnika jakości, parametrów i ograniczeń zawartych w problemie. Jest ona również częścią przygotowanej paczki \texttt{TanksPkg}.

\lstinputlisting[language=modelica, firstline=64, lastline=85]
{../masthe-tango-ds-tanks-regulation/optimica_model/opt_3_tanks.mop}

Ze względu na opisane cechy platformy JModelica.org oraz na fakt, iż stosowano ją do rozwiązywania podobnych problemów (np. \cite{Hast09}), zdecydowano się zastosować ją jako oprogramowanie rozwiązujące rozważany w niniejszej pracy problem sterowania czasooptymalnego. Pozwala ona również na linearyzację modeli, co zostało wykorzystane w celu przygotowania do obliczenia nastaw regulatora liniowo-kwadratowego.

%-------------------------------------------------
\subsection{Środowisko testowe części obliczeniowej}
\label{sub:czesc-wyzsza-docker}

Ze względu na długi i skomplikowany proces instalacji pakietu JModelica.org z jego zależnościami zdecydowano przygotować przenośne środowisko testowe części obliczeniowej opisywanego w niniejszej pracy oprogramowania. W tym celu użyto platformy Docker służącej do zarządzania kontenerami. Są one lekkimi opakowaniami na aplikacje i ich zależności symulującymi zachowanie systemu operacyjnego, które są gotowe do uruchomienia przez silnik Dockera. Do ich zapisu używa się specjalnego systemu kontroli wersji, który zachowuje konkretny stan kontenera (nazywany obrazem), aby można go było w szybki sposób odtworzyć.
Oprócz tego istnieje również możliwość grupowania kontenerów w środowiska pracy, które uruchamiane są jednym poleceniem.

Jest to idealne środowisko do testowania aplikacji wymagających wielu zależności, w tym również sieciowych. Na potrzeby części obliczeniowej przygotowano obraz platformy Docker zawierający:
\begin{itemize}
    \item oprogramowanie JModelica.org wraz z zależnościami:
    \begin{itemize}
        \item pakiet do optymalizacji nieliniowej IPOPT (ang. \emph{Interior Point Optimizer}) oraz jego zależności:
        \begin{itemize}
            \item biblioteka operacji wektorowych i macierzowych BLAS (ang. \emph{Basic Linear Algebra Subprograms}),
            \item biblioteka operacji algebraicznych LAPACK (ang. \emph{Linear Algebra Package}),
            \item biblioteka opisu problemów matematycznych AMPL/MP,
            \item oprogramowanie do rozwiązywania układów liniowych MUMPS;
        \end{itemize}
        \item pakiety do wyliczania wartości pochodnych funkcji CppAD i CasADi,
        \item biblioteka do rozwiązywania równań różniczkowych i algebraicznych Sundials (ang. \emph{Suite of Nonlinear and Differential/Algebraic Equation Solvers}),
        \item pakiet funkcji matematycznych w języku Python SciPy;
    \end{itemize}
    \item oprogramowanie Tango Controls w wersji klienckiej (bez serwera do komunikacji z bazą danych) wraz z zależnościami:
    \begin{itemize}
        \item pakiet implementujący architekturę komunikacji CORBA omniORB,
        \item pakiet do komunikacji zdarzeniowej ZeroMQ,
        \item biblioteka obliczeń numerycznych języka Python NumPy;
    \end{itemize}
    \item pakiet PyTango, czyli API systemu Tango Controls w języku Python.
\end{itemize}

Przygotowano również środowisko pracy zawierające następujące kontenery:
\begin{itemize}
    \item bazę danych z silnikiem MariaDB,
    \item serwer urządzeń DatabaseDS systemu Tango Controls służący jako host systemu,
    \item opisany wyżej kontener służący do uruchamiania części obliczeniowej aplikacji.
\end{itemize}

Takie podejście pozwoliło uniezależnić testowanie tego poziomu aplikacji od posiadanego sprzętu, jeśli tylko obsługiwał on platformę Docker.