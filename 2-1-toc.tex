\section{Regulacja czasooptymalna}
\label{sec:toc}

W niniejszym podrozdziale przedstawiona została koncepcja regulacji czasooptymalnej. Zostały podane założenia zagadnienia, twierdzenia, na których podstawie można wyliczyć rozwiązanie oraz jego proponowana forma analityczna.

Zaznacza się, że mimo iż podane niżej definicje i założenia są wzięte z ogólnych zagadnień optymalizacji dynamicznej, to w podanym brzmieniu stosują się tylko do zagadnienia wyznaczania sterowania czasooptymalnego.

%-------------------------------------------------
\subsection{Ogólna definicja zagadnienia}
\label{sub:toc-def}

%-------------------
\subsubsection{Założenia wstępne}
\label{sub:toc-def-intro}
Dany jest układ opisany stacjonarnym, zwyczajnym równaniem różniczkowym (pokazanym jako równanie \ref{eq:general_system}), w którym:

\begin{equation}\label{eq:general_system}
    \frac{\partial x(t)}{\partial t} = f(x(t), u(t)), ~ 0 \leq t \leq T
\end{equation}

\begin{itemize}
    \item wektor zmiennych stanu w chwili $t$ - $x(t)$ spełnia następujące założenia:
    \begin{itemize}
        \item $\forall_{t \geq 0}:~ x(t) \in \mathbb{R}^{n}$ - ma $n$ składowych, a więc rozważany system ma $n$ równań różniczkowych zwyczajnych,
        \item $x(0) = x_{0} \in \mathbb{R}^{n}$ - spełnia warunek początkowy $x_{0}$;
    \end{itemize}
    \item wektor sterowań w chwili $t$ - $u(t)$:
    \begin{itemize}
        \item $\forall_{t \geq 0}:~ u(t) \in D \subset \mathbb{R}^{m}$ - ma $m$ składowych zawierającym się w zbiorze $D$ ograniczającym wartości sterowań,
        \item $u(0) = u_{0}$ - spełnia warunek początkowy $u_{0}$,
        \item funkcja $u$ jest przedziałami ciągła na przedziale $[0, T]$ (dokładny opis w \cite{Korytowski2015}), czyli:
        \begin{itemize}
            \item ma co najwyżej skończoną liczbę punktów nieciągłości,
            \item w każdym z nich ma skończoną granicę lewostronną,
            \item jest prawostronnie ciągła,
            \item w lewym końcu przedziału jest lewostronnie ciągła;
        \end{itemize}
    \end{itemize}
    \item funkcja $f: \mathbb{R}^{n} \times \mathbb{R}^{m} \longmapsto \mathbb{R}^{n}$:
    \begin{itemize}
        \item $f \in C^{1}$ - jest ciągła i różniczkowalna ze względu na pierwszy argument,
        \item $\frac{\partial f(t)}{\partial x} \in C^{0}$ - jej pochodna ze względu na pierwszy argument jest ciągła.
    \end{itemize}
\end{itemize}

Rozwiązaniem takiego równania jest oczywiście funkcja $x: [0, T) \longmapsto \mathbb{R}^{n}$ nazywana \emph{trajektorią układu}. 

Trajektoria będąca rozwiązaniem zagadnienia minimalnoczasowego musi spełniać następujący warunek końcowy (nazywany również stanem docelowym):
\begin{equation}\label{eq:final_term}
    x(T) = x_{f} \in \mathbb{R}^{n}
\end{equation}

Ów czas $T$, po którym przy danym sterowaniu stan systemu osiągnie warunek końcowy, będzie wskaźnikiem jakości: 
\begin{equation}\label{eq:quality}
    Q(u(t)) = q(x_{f}) = T
\end{equation}

Na tej podstawie można określić \emph{sterowanie optymalne} $\hat{u}(t)$, które spełnia wszystkie wspominane przy opisie równania \ref{eq:general_system} warunki oraz zależność \ref{eq:optimal_quality}. Trajektoria układu wygenerowana przez zastosowanie sterowania optymalnego nazwana jest \emph{trajektorią optymalną} i opisana symbolem $\hat{x}(t)$.
\begin{equation}\label{eq:optimal_quality}
    \forall_{u(t)}:~ Q(u(t)) \leq Q(\hat{u}(t))
\end{equation}

W opisie zagadnienia czasooptymalnego potrzebne są jeszcze dwa pojęcia.
Pierwsze to \emph{funkcja sprzężona} $\psi: [0, T] \longmapsto \mathbb{R}^{n}$ będąca rozwiązaniem tzw. równania sprzężonego \ref{eq:costate-def}.
\begin{equation}\label{eq:costate-def}
\frac{\partial \psi(t)}{\partial t} = - \frac{\partial f(x(t), u(t)}{\partial x}
\end{equation}

Tak, jak w przypadku trajektorii optymalnej układu, trajektoria $\psi(t)$ wyznaczona w układzie, w którym zastosowane zostało sterowanie optymalne $\hat{u}(t)$, nosi nazwę \emph{trajektorii sprzężonej optymalnej} i oznaczona jest symbolem $\hat{\psi}(t)$.

Ostatnim pojęciem potrzebnym w niniejszym zagadnieniu jest \emph{hamiltonian}, zwany również \emph{funkcją Hamiltona}, czyli funkcja $H: \mathbb{R}^{n} \times \mathbb{R}^{n} \times \mathbb{R}^{m} \longmapsto \mathbb{R}$ który dla trajektorii układu $x(t)$ wygenerowanej przy pomocy sterowania $u(t)$ i odpowiadającej im trajektorii sprzężonej $\psi(t)$ zdefiniowany jest zależnością \ref{eq:hamiltonian-def}.

\begin{equation}\label{eq:hamiltonian-def}
H(\psi(t), x(t), u(t)) = \psi(t) \circ f(x(t), u(t)) = \psi(t)^{T} \cdot f(x(t), u(t))
\end{equation}

%-------------------
\subsubsection{Zasada maksimum Pontriagina}
\label{sub:toc-def-pontriagin}

Aby jednoznacznie opisać, a następnie wyznaczyć sterowanie czasooptymalne, potrzebne jest przytoczenie zasadniczego twierdzenia w optymalizacji dynamicznej. Jest ono znane pod nazwą \emph{zasada maksimum Pontriagina}. Zostało opracowane w 1956 r. przez rosyjskiego matematyka Lwa Pontriagina. Twierdzenie podaje się w brzmieniu z \cite{Korytowski2015}.

\begin{pontriagin-max}\label{the:pontryagin}
    Zakładając układ opisany równaniem \ref{eq:general_system} z warunkiem końcowym \ref{eq:final_term} i wskaźnikiem jakości \ref{eq:quality} oraz równanie sprzężone \ref{eq:costate-def}:
    jeśli dla trajektorii układu $\hat{x}(t)$ wygenerowanej przy pomocy sterowania $\hat{u}(t)$ i odpowiadającej im trajektorii sprzężonej $\hat{\psi}(t)$ zachodzi:
    \begin{equation}\label{eq:pontriagin}
    \forall_{u(t) \in D}~ \forall_{t \in [0, T]}:~ H(\hat{\psi}(t), \hat{x}(t), \hat{u}(t)) ~ \geq ~ H(\hat{\psi}(t), \hat{x}(t), u(t))
    \end{equation}
    to sterowanie $\hat{u}(t)$ jest sterowaniem optymalnym.
\end{pontriagin-max}

Dowód zasady maksimum nie został uwzględniony w tej pracy. Można go znaleźć w \cite{Korytowski2015} oraz w \cite{Evans}.

Powyższe twierdzenie należy obwarować dodatkowymi warunkami koniecznymi optymalności. Niech funkcja $g: \mathbb{R}^{2n} \longmapsto \mathbb{R}^{l}$ opisuje zestaw ograniczeń nierównościowych, a funkcja $h: \mathbb{R}^{2n} \longmapsto \mathbb{R}^{k}$ - ograniczeń równościowych. Obie dane są wzorem \ref{eq:pontryagin-constraints}. Dodatkowo zakłada się, że $g, h \in C^{1}$.
\begin{equation}\label{eq:pontryagin-constraints}
    \begin{array}{lr}
    g(x_{0}, x_{f}) \leq 0 \\
    h(x_{0}, x_{f}) = 0
    \end{array}
\end{equation}

Ponadto, zakłada się, że istnieją $\lambda \in \mathbb{R}$, $\mu \in \mathbb{R}^{l}$ oraz $\rho \in \mathbb{R}^{k}$, które wraz z uprzednio zdefiniowanymi wielkościami i funkcjami spełniają następujące \emph{warunki konieczne optymalności}:
\begin{itemize}
    \item warunek nieujemności:
    \begin{equation}\label{eq:pontryagin-noneg}
    \lambda \geq 0 \land \|\mu\| \geq 0
    \end{equation}
    \item warunek nietrywialności:
    \begin{equation}\label{eq:pontryagin-notriv}
    \lambda + \|\mu\| + \|\rho\| > 0
    \end{equation}
    \item warunek komplementarności:
    \begin{equation}\label{eq:pontryagin-comp}
    \mu \circ g(x_{0}, x_{f}) = 0
    \end{equation}
    \item warunki transwersalności:
    \begin{equation}\label{eq:pontryagin-trans}
    \begin{array}{lr}
        \hat{\psi}(0) = \frac{\partial (\mu \circ g + \rho \circ h)}{\partial x_{0}}\\[8pt]
        \hat{\psi}(T) = - \frac{\partial (\mu \circ g + \rho \circ h)}{\partial x_{f}}\\[8pt]
        \forall_{t \in [0, T]}:~ H(\psi(t), x(t), u(t)) = \frac{\partial (\lambda T + \mu \circ g + \rho \circ h)}{\partial T} = \lambda
    \end{array}
    \end{equation}
    \item równanie sprzężone dane wzorem \ref{eq:costate-def},
    \item warunek maksimum hamiltonianu dany nierównością \ref{eq:pontriagin}.
\end{itemize}


%-------------------------------------------------
\subsection{Nieliniowość układu a sterowanie czasooptymalne}
\label{sub:toc-nonlnr}

Poszukiwanie sterowania czasooptymalnego w systemach nieliniowych jest w ogólności bardzo skomplikowane, przede wszystkim ze względu na trudność rozwiązania analitycznego nieliniowych (lub nawet niestacjonarnych) równań różniczkowych. Układ zbiorników omawiany w niniejszej pracy również takie posiada: zarówno równania dynamiki układu \ref{eq:model}, jak i równania sprzężone \ref{eq:model-costate} są nieliniowe.

Aby uprościć analizę problemu oraz fizyczne zastosowanie wyznaczonego sterowania, zakłada się, że poszukiwane jego postaci będą klasy ,,bang - bang''. To znaczy, że będą przyjmowały tylko wartości z brzegów jednowymiarowego zbioru dopuszczalnego $D \in \mathbb{R} \land D = [0, u_{max}] \Rightarrow \forall_{t \in [0, T]}:~ u(t) \in [0, u_{max}]$.

\begin{equation}\label{eq:nonlin-bang-bang}
    \frac{\partial x}{\partial t} = f(x(t)) + \sum_{i=1}^{m} g_{i}(x(t)) \cdot u_{i}(t)
\end{equation}

Można przyjąć takie założenie, ze względu na to, iż (jak pokazano w \cite{YiMa2008} oraz w rozdziale 7.10 \cite{AthansOptCtrl}) dla systemów, których równania mają postać opisaną równaniem \ref{eq:nonlin-bang-bang} (a taką dokładnie ma omawiany układ z $m = 1$), funkcja przełączająca ma wtedy postać: $\phi(t) = \hat{\psi}(t) \circ g(\hat{x}(t))$. Owa funkcja opisuje momenty, w których sterowanie zmienia swoją wartość z jednego krańca zbioru $D$ na drugi. W przypadku rozważanego układu zbiorników sterowanie optymalne będzie dane wzorem \ref{eq:model-opt-ctrl}.

\begin{equation}\label{eq:model-opt-ctrl}
\begin{array}{lr}
    \hat{u}(t) = \frac{sgn(\phi(t)) + sgn(\phi(t))^{2}}{2} \cdot u_{max} ~ \land ~ \phi(t) = \frac{\hat{\psi}_{1}(t)}{aw} \Rightarrow \\
    \hat{u}(t) = \frac{sgn(\hat{\psi}_{1}(t) + sgn(\hat{\psi}_{1}(t))^{2}}{2} \cdot u_{max}
\end{array}
\end{equation}

Dodatkowo, jak wspomniano w podrozdziale \ref{sec:model}, układ otwarty jest stabilny, a ograniczone sterowanie nie może tego zmienić.

Podobne założenia są częstą praktyką w analizie stabilnych systemów nieliniowych ze względu na prostotę fizycznej aplikacji sterowania ,,bang - bang''. Przykłady znajdują się m.in. w \cite{VakKek82}, \cite{BalSom83} oraz \cite{Itik2016}.

%-------------------------------------------------
\subsection{Analityczne metody wyznaczania sterowania czasooptymalnego}
\label{sub:toc-ctrl}

Analityczne rozwiązania poszukiwania sterowania czasooptymalnego zwykle opierają się bezpośrednio na przytoczonej powyżej zasadzie maksimum i warunkach koniecznych optymalności. W niniejszym podrozdziale zostanie pokrótce przedstawiona droga mogąca zmierzać do wyznaczenia analitycznego czasooptymalnego sterowania w rozważanym układzie zbiorników. Całe rozwiązanie nie jest przeprowadzone ze względu na fakt, iż równania sprzężone są niestacjonarne, a więc ich rozwiązanie analityczne byłoby bardzo trudne lub wręcz niemożliwe.

Pierwszym krokiem ku wyliczeniu analitycznego rozwiązania jest wyznaczenie równań sprzężonych za pomocą wzoru \ref{eq:costate-def}. Przyjmując współczynniki $\alpha_{i} = \frac{1}{2} \forall_{i \in \{1, 2, 3\}}$ w modelu matematycznym zestawu zbiorników danym równaniem \ref{eq:model}, można wyznaczyć równania sprzężone rozważanego układu. Są one dane wzorem \ref{eq:model-costate}. Pominięto w nim zależności wszystkich funkcji $\psi$ oraz $h$ od czasu, aby uprościć zapis.

\begin{equation}\label{eq:model-costate}
	\left \{
	\begin{array}{lr}
		\frac{\partial \psi_{1}}{\partial t} =  \psi_{1}\frac{C_{1}}{2aw\sqrt{h_{1}}} - \psi_{2}\frac{C_{1}}{2\sqrt{h_{1}}(cw + \frac{h_{2}}{h_{max}}bw)} \\[20pt]
		\frac{\partial \psi_{2}}{\partial t} = - \psi_{2}\frac{1}{cw + \frac{h_{2}}{h_{max}}bw}(\frac{b(C_{1}\sqrt{h_{1}} - C_{2}\sqrt{h_{2}})}{ch_{max} + bh_{2}} - \frac{C_{2}}{2\sqrt{h_{2}}}) - \psi_{3}\frac{1}{w\sqrt{h_{3}(2R - h_{3})}} \\[20pt]
		\frac{\partial \psi_{3}}{\partial t} = \psi_{3}\frac{-C_{3}(3R - 2h_{3})}{wh_{3}(2R - h_{3})^{\frac{3}{2}}}
	\end{array}
	\right.
\end{equation}

Następnie trzeba by przedefiniować ograniczenia równościowe (dane wzorem \ref{eq:model-eq-const}) i nierównościowe (\ref{eq:model-noneq-const}) tak, aby spełniały założenia funkcji $g$ i $h$ opisane zależnościami \ref{eq:pontryagin-constraints}.
Dodatkowo należy dopisać od ograniczeń równościowych to wynikające z faktu poszukiwania sterowania czasooptymalnego, czyli \ref{eq:final_term}, które w tym przypadku będzie miało postać opisaną przez \ref{eq:model-final-term}, gdzie $h_{1f}$, $h_{2f}$ i $h_{3f}$ są dane.

\begin{equation}\label{eq:model-final-term}
\begin{array}{lr}
    h_{1}(T) = h_{1f}\\
    h_{2}(T) = h_{2f}\\
    h_{3}(T) = h_{3f}
\end{array}
\end{equation}

Korzystając z warunków komplementarności (\ref{eq:pontryagin-comp}) oraz nieujemności (\ref{eq:pontryagin-noneg}), powinno się wyznaczyć składowe wektora $\mu$ oraz założyć pewną postać wektora $\rho$ (bazując takie założenie na \ref{eq:pontryagin-notriv}).

Na podstawie tych danych należałoby wyznaczyć warunki początkowy i końcowy danym wzorem \ref{eq:pontryagin-trans} dla powyższych równań sprzężonych, co pozwoliłoby wyznaczyć analityczne wzory opisujące wszystkie składowe trajektorii sprzężonych systemu.

Na koniec, korzystając z zależności \ref{eq:model-opt-ctrl}, można by wyznaczyć analityczny wzór sterowania optymalnego.


%-------------------------------------------------
\subsection{Numeryczne metody wyznaczania sterowania czasooptymalnego}
\label{sub:toc-num}

W niniejszej sekcji zostały przekrojowo zaprezentowane numeryczne metody poszukiwania sterowania czasooptymalnego dla układów nieliniowych.
Nie jest jej celem przedstawiać dokładne działanie wszystkich istniejących algorytmów, ale raczej dokonać klasyfikacji i naświetlić interesujące metody oraz odesłać do odpowiedniej literatury przedmiotu (przede wszystkim \cite{Betts98}).

Podstawowy podział metod obliczeniowych to metody pośrednie oraz bezpośrednie.

% TODO: Napisać podsekcję o numerycznych metodach wyznaczania sterowania!