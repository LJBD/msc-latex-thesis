\section{Model matematyczny zestawu zbiorników}
\label{sec:model}

Na podstawie podanych wcześniej zależności można zdefiniować model matematyczny rozważanego układu zbiorników.
Jest on dany równaniem \ref{eq:model}.

\begin{equation}\label{eq:model}
\left \{
\begin{array}{lr}
\frac{\partial h_{1}}{\partial t} = \frac{u - C_{1}{h_{1}}^{\alpha_{1}}}{aw} \\[8pt]
\frac{\partial h_{2}}{\partial t} = \frac{C_{1}{h_{1}}^{\alpha_{1}} -  C_{2}{h_{2}}^{\alpha_{2}}}{cw + \frac{h_{2}}{h_{max}}bw} \\[20pt]
\frac{\partial h_{3}}{\partial t} = \frac{C_{2}{h_{2}}^{\alpha_{2}} -  C_{3}{h_{3}}^{\alpha_{3}}}{w\sqrt{R^{2} - (R - h_{3})^{2}}}
\end{array}
\right.
\end{equation}

Oznaczenia:
\begin{itemize}
    \item $h_{i}(t)$ - poziom wody w $i$-tym zbiorniku ($i \in \{1, 2, 3\}$),
    \item $u(t)$ - sterowanie pompą,
    \item $a$ - szerokość pierwszego zbiornika,
    \item $b$ - szerokość trójkątnej części drugiego zbiornika,
    \item $c$ - szerokość prostopadłościennej części drugiego zbiornika,
    \item $R$ - promień trzeciego zbiornika,
    \item $w$ - głębokość zbiorników,
    \item $h_{max}$ maksymalna wysokość słupa wody w zbiornikach,
    \item $C_{i}$ - opór wypływu z $i$-tego zbiornika,
    \item $\alpha_{i}$ - współczynnik wypływu z $i$-tego zbiornika.
\end{itemize}

Wszystkie wymiary w powyższym wzorze zostały przedstawione na rys. \ref{fig:zbiorniki}. Są na nim również oznaczone opory wypływów $C_{1}$ - $C_{3}$ przy odpowiednich zaworach.

Przyjmując $\alpha_{i} = \frac{1}{2}, \forall_{i \in \{1, 2, 3\}}$, można uszczegółowić powyższy model, zakładając tylko przepływ laminarny.
Taka właśnie jego postać będzie wykorzystywana przy analitycznym wyznaczeniu współczynników równania sprzężonego (definicja znajduje się w sekcji \ref{sub:toc-def-intro}), które jest przeprowadzone w podrozdziale \ref{sub:toc-ctrl}.
W rzeczywistości, jak zostało wspomniane w podrozdziale \ref{sub:plyny-przeplywy}, wartości tych współczynników będą musiały być trochę mniejsze, aby oddać faktyczny sposób przepływu wody między zbiornikami.

%TODO: dopisać definicję wszystkich ograniczeń
W rozważanym układzie zbiorników przyjmuje się następujące ograniczenia:
\begin{itemize}
    \item ograniczenia równościowe:
    \begin{equation}\label{eq:model-eq-const}
    \begin{array}{lr}
        h_{1}(0) = h_{10}\\
        h_{2}(0) = h_{20}\\
        h_{3}(0) = h_{30}\\
    \end{array}
    \end{equation}
    \item ograniczenia nierównościowe:
    \begin{equation}\label{eq:model-noneq-const}
    \begin{array}{lr}
        \forall_{t \in [0, T]}:~ 0 \leq h_{1}(t) \leq h_{max}\\
        \forall_{t \in [0, T]}:~ 0 \leq h_{2}(t) \leq h_{max}\\
        \forall_{t \in [0, T]}:~ 0 \leq h_{3}(t) \leq h_{max}\\
        \forall_{t \in [0, T]}:~ 0 \leq u(t) \leq u_{max}
    \end{array}
    \end{equation}
\end{itemize}
