\section{Model matematyczny zestawu zbiorników}
\label{sec:model}


\begin{equation}\label{eq:model}
	\left \{
	\begin{array}{lr}
		\frac{\partial h_{1}}{\partial t} = \frac{u - C_{1}{h_{1}}^{\alpha_{1}}}{aw} \\[8pt]
		\frac{\partial h_{2}}{\partial t} = \frac{C_{1}{h_{1}}^{\alpha_{1}} -  C_{2}{h_{2}}^{\alpha_{2}}}{cw + \frac{h_{2}}{h_{max}}bw} \\[20pt]
		\frac{\partial h_{3}}{\partial t} = \frac{C_{2}{h_{2}}^{\alpha_{2}} -  C_{3}{h_{3}}^{\alpha_{3}}}{w\sqrt{R^{2} - (R - h_{3})^{2}}}
	\end{array}
	\right.
\end{equation}

Model matematyczny zestawu zbiorników jest dany równaniem \ref{eq:model}, gdzie:
\begin{itemize}
	\item $a$ - szerokość pierwszego zbiornika,
	\item $b$ - szerokość trójkątnej części drugiego zbiornika,
	\item $c$ - szerokość prostopadłościennej części drugiego zbiornika,
	\item $R$ - promień trzeciego zbiornika,
	\item $w$ - głębokość zbiorników,
	\item $h_{max}$ maksymalna wysokość słupa wody w zbiornikach,
	\item $C_{1}$ - opór wypływu z pierwszego zbiornika,
	\item $C_{2}$ - opór wypływu z drugiego zbiornika,
	\item $C_{3}$ - opór wypływu z trzeciego zbiornika,
    \item $\alpha_{1}$ - współczynnik wypływu z pierwszego zbiornika,
    \item $\alpha_{2}$ - współczynnik wypływu z drugiego zbiornika,
    \item $\alpha_{3}$ - współczynnik wypływu z trzeciego zbiornika.
\end{itemize}
Wszystkie wymiary w powyższym wzorze zostały przedstawione na rys. \ref{fig:zbiorniki}.

