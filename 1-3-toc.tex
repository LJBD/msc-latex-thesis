\section{Regulacja czasooptymalna}
\label{sec:toc}

W niniejszym podrozdziale przedstawiona zostanie koncepcja regulacji czasooptymalnej. Zostaną podane założenia zagadnienia, twierdzenia, na których podstawie można wyliczyć rozwiązania oraz jego proponowana forma analityczna.

%-------------------------------------------------
\subsection{Ogólna definicja zagadnienia}
\label{sub:toc-def}


\subsubsection{Założenia wstępne}
\label{sub:toc-def-intro}
Przyjmijmy system dany stacjonarnym, zwyczajnym równaniem różniczkowym (pokazanym jako równanie \ref{eq:general_system}), w którym:

\begin{equation}\label{eq:general_system}
    \frac{\partial x(t)}{\partial t} = f(x(t), u(t)), ~ 0 \leq t \leq T
\end{equation}

\begin{itemize}
    \item wektor zmiennych stanu w chwili $t$ - $x(t)$ spełnia następujące założenia:
    \begin{itemize}
        \item $\forall_{t \geq 0}:~ x(t) \in \mathbb{R}^{n}$ - ma $n$ składowych, a więc rozważany system ma $n$ równań różniczkowych zwyczajnych,
        \item $x(0) = x_{0} \in \mathbb{R}^{n}$ - spełnia warunek początkowy $x_{0}$;
    \end{itemize}
    \item wektor sterowań w chwili $t$ - $u(t)$:
    \begin{itemize}
        \item $\forall_{t \geq 0}:~ u(t) \in D \subset \mathbb{R}^{m}$ - ma $m$ składowych zawierającym się w zbiorze $D$ ograniczającym wartości sterowań,
        \item $u(0) = u_{0}$ - spełnia warunek początkowy $u_{0}$,
        \item funkcja $u$ jest przedziałami ciągła na przedziale $[0, T]$, czyli:
        \begin{itemize}
            \item ma co najwyżej skończoną liczbę punktów nieciągłości,
            \item w każdym z nich ma skończoną granicę lewostronną,
            \item jest prawostronnie ciągła,
            \item w lewym końcu przedziału jest lewostronnie ciągła;
        \end{itemize}
    \end{itemize}
    \item funkcja $f: \mathbb{R}^{n} \times \mathbb{R}^{m} \longmapsto \mathbb{R}^{n}$:
    \begin{itemize}
        \item $f \in C^{1}$ - jest ciągła i różniczkowalna ze względu na pierwszy argument,
        \item $\frac{\partial f(t)}{\partial x} \in C^{0}$ - jej pochodna ze względu na pierwszy argument jest ciągła.
    \end{itemize}
\end{itemize}

Rozwiązaniem takiego równania jest oczywiście funkcja $x: [0, T) \longmapsto \mathbb{R}^{n}$ nazywana \emph{trajektorią układu}.

Trajektoria będąca rozwiązaniem zagadnienia minimalnoczasowego musi spełniać następujący warunek końcowy (nazywany również stanem docelowym):
\begin{equation}\label{eq:final_term}
    x(T) = x_{f} \in \mathbb{R}^{n}
\end{equation}

Ów czas $T$, po którym przy danym sterowaniu stan systemu osiągnie warunek końcowy, będzie wskaźnikiem jakości: 
\begin{equation}\label{eq:quality}
    Q(u) = T
\end{equation}

Na tej podstawie można określić \emph{sterowanie optymalne} $\hat{u}(t)$, które spełnia wszystkie wspominane wcześniej warunki oraz zależność \ref{eq:optimal_quality}. Trajektoria układu wygenerowana przez zastosowanie sterowania optymalnego nazwana jest \emph{trajektorią optymalną} i opisana symbolem $\hat{x}(t)$.
\begin{equation}\label{eq:optimal_quality}
    \forall_{u}:~ Q(u) \leq Q(\hat{u})
\end{equation}

\subsubsection{Zasada maksimum Pontriagina}
\label{sub:toc-def-pontriagin}


%-------------------------------------------------
\subsection{Wyznaczanie sterowania czasooptymalnego}
\label{sub:toc-ctrl}


\begin{equation}\label{eq:sprzezone}
	\left \{
	\begin{array}{lr}
		\frac{\partial \psi_{1}}{\partial t} =  \psi_{1}\frac{C_{1}}{2aw\sqrt{h_{1}}} - \psi_{2}\frac{C_{1}}{2\sqrt{h_{1}}(cw + \frac{h_{2}}{h_{max}}bw)} \\[20pt]
		\frac{\partial \psi_{2}}{\partial t} = - \psi_{2}\frac{1}{cw + \frac{h_{2}}{h_{max}}bw}(\frac{b(C_{1}\sqrt{h_{1}} - C_{2}\sqrt{h_{2}})}{ch_{max} + bh_{2}} - \frac{C_{2}}{2\sqrt{h_{2}}}) - \psi_{3}\frac{1}{w\sqrt{h_{3}(2R - h_{3})}} \\[20pt]
		\frac{\partial \psi_{3}}{\partial t} = \psi_{3}\frac{-C_{3}(3R - 2h_{3})}{wh_{3}(2R - h_{3})^{\frac{3}{2}}}
	\end{array}
	\right.
\end{equation}

Równania sprzężone zestawu zbiorników (przy założeniu przepływu laminarnego) są dane wzorem \ref{eq:sprzezone}.


%-------------------------------------------------
\subsection{Nieliniowość układu a sterowanie czasooptymalne}
\label{sub:toc-nonlnr}


%-------------------------------------------------
\subsection{Numeryczne metody wyznaczania sterowania czasooptymalnego}
\label{sub:toc-num}