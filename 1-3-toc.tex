\section{Regulacja czasooptymalna}
\label{sec:toc}

%-------------------------------------------------
\subsection{Ogólna definicja zagadnienia}
\label{sub:toc-def}


%-------------------------------------------------
\subsection{Wyznaczanie sterowania czasooptymalnego}
\label{sub:toc-ctrl}


\begin{equation}\label{eq:sprzezone}
	\left \{
	\begin{array}{lr}
		\frac{\partial \psi_{1}}{\partial t} =  \psi_{1}\frac{C_{1}}{2aw\sqrt{h_{1}}} - \psi_{2}\frac{C_{1}}{2\sqrt{h_{1}}(cw + \frac{h_{2}}{h_{max}}bw)} \\[20pt]
		\frac{\partial \psi_{2}}{\partial t} = - \psi_{2}\frac{1}{cw + \frac{h_{2}}{h_{max}}bw}(\frac{b(C_{1}\sqrt{h_{1}} - C_{2}\sqrt{h_{2}})}{ch_{max} + bh_{2}} - \frac{C_{2}}{2\sqrt{h_{2}}}) - \psi_{3}\frac{1}{w\sqrt{h_{3}(2R - h_{3})}} \\[20pt]
		\frac{\partial \psi_{3}}{\partial t} = \psi_{3}\frac{-C_{3}(3R - 2h_{3})}{wh_{3}(2R - h_{3})^{\frac{3}{2}}}
	\end{array}
	\right.
\end{equation}

Równania sprzężone zestawu zbiorników (przy założeniu przepływu laminarnego) są dane wzorem \ref{eq:sprzezone}.


%-------------------------------------------------
\subsection{Nieliniowość układu a sterowanie czasooptymalne}
\label{sub:toc-nonlnr}


%-------------------------------------------------
\subsection{Numeryczne metody wyznaczania sterowania czasooptymalnego}
\label{sub:toc-num}