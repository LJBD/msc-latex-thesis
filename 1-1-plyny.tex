\section{Wprowadzenie z zakresu dynamiki płynów}
\label{sec:plyny}

W niniejszym podrozdziale zostaną przypomniane podstawowe prawa fizyki związane w przepływem cieczy oraz jego związkiem z jej poziomem w zbiorniku.

%-------------------------------------------------
\subsection{Równanie Bernoulliego i prawo Torricellego}
\label{sub:plyny-torr}

Równanie Bernoulliego jest jednym z podstawowych praw termodynamiki płynów idealnych. Mówi ono, że wzrost prędkości przepływu cieczy musi wiązać się ze spadkiem ciśnienia lub energii potencjalnej. Ma kilka postaci - najpopularniejszą jest tzw. szczególne równanie Bernoulliego, które wiąże energię mechaniczną płynu z jego prędkością w danym miejscu, wysokością w układzie odniesienia służącym do wyznaczania energii potencjalnej, ciśnieniem i gęstością. W takiej formie można je stosować tylko do cieczy nieściśliwych i nielepkich, jednocześnie zakładając stacjonarność i bezwirowość przepływu.

Ta szczególna postać równania Bernoulliego jest przedstawiona jako równanie \ref{eq:bernoulli}, gdzie:
\begin{itemize}
	\item $e_{m}$ - energia jednostki masy cieczy,
	\item $v$ - prędkość cieczy w danym miejscu,
	\item $g$ - przyspieszenie grawitacyjne,
	\item $h$ - wysokość w układzie odniesienia, w którym jest wyznaczana energia potencjalna,
	\item $p$ - ciśnienie cieczy w danym miejscu,
	\item $\rho$ - gęstość cieczy.
\end{itemize}

\begin{equation}\label{eq:bernoulli}
	e_{m} = \frac{v^2}{2} + gh + \frac{p}{\rho} = const
\end{equation}

Z równania Bernoulliego można wyprowadzić bezpośrednią zależność między prędkością cieczy a jej poziomem w zbiorniku. Jest ona znana pod nazwą prawa Torricellego i przedstawiona jako równanie \ref{eq:torricelli} (przyjęto oznaczenia takie, jak w przypadku równania \ref{eq:bernoulli}).

\begin{equation}\label{eq:torricelli}
	v = \sqrt{2gh}
\end{equation}

Można owo prawo zapisać w bardziej ogólnej formie słownej: prędkość wypływu cieczy jest proporcjonalna do pierwiastka z poziomu cieczy w zbiorniku. Takie jego sformułowanie będzie istotne przy wyznaczaniu modelu matematycznego rozważanego układu.


%-------------------------------------------------
\subsection{Przepływ laminarny i nielaminarny}
\label{sub:plyny-przeplyw}