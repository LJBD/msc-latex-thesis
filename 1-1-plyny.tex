\section{Wprowadzenie z zakresu dynamiki płynów}
\label{sec:plyny}

W niniejszym podrozdziale zostały przypomniane podstawowe prawa fizyki związane z przepływem cieczy oraz jego związkiem z poziomem tej cieczy w zbiorniku.

%-------------------------------------------------
\subsection{Równanie Bernoulliego i prawo Torricellego}
\label{sub:plyny-torr}

Równanie Bernoulliego jest jednym z podstawowych praw termodynamiki płynów idealnych. Mówi ono, że wzrost prędkości przepływu cieczy musi wiązać się ze spadkiem ciśnienia lub energii potencjalnej.
Ma kilka postaci; najpopularniejszą jest tzw. szczególne równanie Bernoulliego, które wiąże energię mechaniczną płynu z jego prędkością w danym miejscu, wysokością w układzie odniesienia służącym do wyznaczania energii potencjalnej, ciśnieniem i gęstością.
W takiej formie można je stosować tylko do cieczy nieściśliwych i nielepkich, jednocześnie zakładając stacjonarność i bezwirowość przepływu.

Ta szczególna postać równania Bernoulliego jest przedstawiona jako równanie \ref{eq:bernoulli}.

\begin{equation}\label{eq:bernoulli}
e_{m} = \frac{v^2}{2} + gh + \frac{p}{\rho} = const
\end{equation}

Oznaczenia:
\begin{itemize}
	\item $e_{m}$ - energia jednostki masy cieczy,
	\item $v$ - prędkość cieczy w danym miejscu,
	\item $g$ - przyspieszenie grawitacyjne,
	\item $h$ - wysokość w układzie odniesienia, w którym jest wyznaczana energia potencjalna,
	\item $p$ - ciśnienie cieczy w danym miejscu,
	\item $\rho$ - gęstość cieczy.
\end{itemize}

Z równania Bernoulliego można wyprowadzić bezpośrednią zależność między prędkością cieczy a jej poziomem w zbiorniku. Jest ona znana pod nazwą prawa Torricellego i przedstawiona jako równanie \ref{eq:torricelli} (przyjęto oznaczenia takie jak w przypadku równania \ref{eq:bernoulli}).
Można owo prawo zapisać w bardziej ogólnej formie słownej:

\begin{torricelli}
    Prędkość wypływu cieczy jest proporcjonalna do pierwiastka kwadratowego z poziomu cieczy w zbiorniku.
    \begin{equation}\label{eq:torricelli}
    v = \sqrt{2gh}
    \end{equation}
\end{torricelli}
Takie sformułowanie tego prawa będzie istotne w dalszych krokach wyznaczania modelu matematycznego rozważanego układu.


\subsection{Bilans masy}
\label{sub:plyny-bilans}

Kolejnym zjawiskiem fizycznym, którego zrozumienie jest potrzebne, aby wyznaczyć model matematyczny rozważanego w niniejszej pracy układu zbiorników, jest bilans masy, czyli bezpośrednia konsekwencja \emph{prawa zachowania masy}.

\begin{mass}
    Masa układu ciał (suma mas wszystkich ciał wchodzących w skład tego układu) nie zmienia się podczas przemian i oddziaływań fizycznych w nim zachodzących.
    \begin{equation}\label{eq:mass-conservation}
        m_{uk} = const
    \end{equation}
\end{mass}

Rozważając układ pojedynczego zbiornika z cieczą, do którego ta ciecz jest nalewana i z którego się ona wylewa, można sformułować następstwo tego prawa dane równaniem \ref{eq:mass-balance}. Mówi ono, że zmiana masy w rozważanym zbiorniku - $m_{zb}$ - jest równa zmianie masy do niego wpływającej - $m_{we}$ - i wypływającej - $m_{wy}$.

\begin{equation}\label{eq:mass-balance}
    \frac{\partial m_{we}}{\partial t} - \frac{\partial m_{wy}}{\partial t} =\frac{\partial m_{zb}}{\partial t}
\end{equation}

Przyjmując założenie, że ciecz w zbiorniku i poza nim ma stałą gęstość $\rho$ oraz stosując następujące podstawienia:
\begin{itemize}
    \item $m_{zb} = V_{zb}\cdot\rho$, gdzie $V_{zb}$ to objętość cieczy w zbiorniku,
    \item $V_{zb} = A_{zb} \cdot h_{zb}$, gdzie $A_{zb}$ to pole przekroju poprzecznego zbiornika, a $h_{zb}$ to wysokość słupa cieczy w tym zbiorniku
\end{itemize}
można przedstawić powyższą zależność w postaci opisanej zależnością \ref{eq:tank-mass-balance} (za: \cite{Postlethwaite}).

\begin{equation}\label{eq:tank-mass-balance}
    A_{zb} \cdot \frac{\partial h_{zb}}{\partial t} = \frac{\partial V_{we}}{\partial t} - \frac{\partial V_{wy}}{\partial t}
\end{equation}

Strumień (zmiana objętości cieczy w czasie) wypływający z takiego zbiornika można otrzymać na podstawie prawa Torricellego - jest on dany zależnością \ref{eq:tank-outflow}, gdzie $C$ to stała proporcjonalności wypływu. W rozważanym układzie będzie on zależeć od ustawienia zaworu wyjściowego z danego zbiornika, a więc można powiedzieć, że strumień opisuje opór wypływu ze zbiornika (za: \cite{TanksManual}).

\begin{equation}\label{eq:tank-outflow}
\frac{\partial V_{wy}}{\partial t} = C\cdot\sqrt{h_{zb}}
\end{equation}

Jeśli chodzi o strumień wpływający, to dla drugiego i trzeciego zbiornika jest on równy strumieniowi wypływającemu z poprzedniego zbiornika. Można przyjąć, że dla pierwszego zbiornika ten strumień to sterowanie pompą. Będzie ono oznaczone symbolem $u$.

Pola powierzchni przekrojów poprzecznych wszystkich trzech zbiorników przedstawiono jako równanie \ref{eq:model-fields}. Zastosowano oznaczenia z \ref{fig:zbiorniki}.

\begin{equation}\label{eq:model-fields}
    \begin{array}{lr}
        A_{1} = w \cdot a \\
        A_{2} = c\cdot w + \frac{h_{2}}{h_{max}}\cdot b\cdot w \\
        A_{3} = w\cdot \sqrt{R^{2} - (R - h_{3})^{2}}
    \end{array}
\end{equation}

%-------------------------------------------------
\subsection{Rodzaje przepływów}
\label{sub:plyny-przeplywy}

Przytoczona wcześniej szczególna postać równania Bernoulliego (równanie \ref{eq:bernoulli}) jest obwarowana założeniem stacjonarności przepływu. Oznacza to dwie rzeczy:
\begin{enumerate}
    \item Wartości wektorów prędkości cieczy są stałe w czasie.
    \item Poszczególne ,,warstwy'' cieczy nie wpływają na siebie.
\end{enumerate}

Drugi z tych warunków jest znany pod nazwą przepływu laminarnego, który zwykle ma miejsce przy niskich prędkościach cieczy. W takim typie przepływu nie występują żadne jego zaburzenia (ruchy wirowe, prądy przeciwne itp.), a zachowanie poszczególnych ,,warstw'' cieczy porównać można do tasowania kart: przepływają obok siebie bez wpływania jedna na drugą. Jej cząstki będące blisko powierzchni przemieszczają się po liniach równoległych do tafli cieczy.

Niestety, w rzeczywistości ciężko jest spełnić założenie laminarności przepływu, nie mówiąc już o jego stacjonarności. W związku z tym można zastosować pewne praktyczne uogólnienie zależności \ref{eq:tank-outflow} w stosunku do cieczy wypływających w sposób nielaminarny. Polega ono na zastąpieniu pierwiastka we wspomnianym wzorze parametrem $\alpha$, którego wartość można dobrać na podstawie pomiarów w rzeczywistym układzie (przykład podany w \cite{TanksManual}). Uwzględniają to uogólnienie, można zapisać nowe sformułowanie zależności \ref{eq:tank-outflow} jako równanie \ref{eq:tank-outflow-nonlmnr}.

\begin{equation}\label{eq:tank-outflow-nonlmnr}
    \frac{\partial V_{wy}}{\partial t} = C\cdot h_{zb}^{\alpha}
\end{equation}
