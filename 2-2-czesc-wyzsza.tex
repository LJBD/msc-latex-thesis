\section{Część obliczeniowa (,,wyższego poziomu'')}
\label{sec:czesc-wyzsza}

Aby zapewnić jak najlepszą realizację zadań postawionych przed częścią obliczeniową, należało wybrać odpowiednie oprogramowanie stanowiące podstawę tego poziomu aplikacji. Założono, że ta część powinna być napisana w języku programowania Python oraz używać 


%-------------------------------------------------
\subsection{Wybór pakietu do optymalizacji dynamicznej}
\label{sub:czesc-wyzsza-wybor}

Podstawowym elementem istotnym dla powodzenia takiego zadania był wybór pakietu realizującego operacje optymalizacji dynamicznej. Istnieje na rynku kilka płatnych bibliotek bądź pakietów oprogramowania, które są w stanie rozwiązywać tego typu problemy (np. MATLAB, Mathematica itp.), ale postanowiono poszukać darmowego odpowiednika, który posiadałby interfejs w języku programowania Python. Takie założenie zostało poczynione ze względu na to, iż biblioteką stanowiącą 


%-------------------------------------------------
\subsection{Opis pakietu Modelica}
\label{sub:czesc-wyzsza-modelica}


%-------------------------------------------------
\subsection{Opis systemu Tango Controls}
\label{sub:czesc-wyzsza-tango}


%-------------------------------------------------
\subsection{Architektura klasy urządzeń systemu Tango}
\label{sub:czesc-wyzsza-klasa}


\subsubsection{Interfejs dla systemu Tango}
%Lista komend, właściwości i atrybutów


\subsubsection{Problem dostępności interfejsu}

%Wątki vs procesy w Pythonie
%Propozycja uniwersalnego rozwiązania długich funkcji w DS Tango


%-------------------------------------------------
\subsection{Środowisko testowe części obliczeniowej}
\label{sub:czesc-wyzsza-docker}