\chapter{Architektura zaproponowanego rozwiązania}
\label{cha:arch}

W niniejszej pracy została zaproponowana hybrydowa struktura aplikacji realizującej zadania wyliczania i aplikowania sterowania czasooptymalnego, jak i minimalnego w sensie liniowo-kwadratowego wskaźnika jakości.
Taka jej architektura jest odbiciem faktycznej tendencji w automatyce ostatnich lat: aby skomplikowane zadania obliczeniowe zadawać nie tym elementom, które realizują bezpośrednie sterowanie, ale zlecać je innym urządzeniom o architekturze sprzętu odpowiedniejszej do radzenia sobie z takimi problemami.

Dokładnie takie podejście zostało również zaprezentowane w niniejszej pracy. Hybrydowa (z łaciny: \emph{hybrida}: mieszaniec, krzyżówka) struktura polega na zastosowaniu dwóch poziomów: obliczeniowego i realizującego sterowanie bezpośrednie. Zostały one szerzej opisane w tym rozdziale.

\section{Podział zadań między elementami oprogramowania}
\label{sec:podzial-zadan}

W związku z zaproponowaną w niniejszej pracy architekturą oprogramowania, opierającą się na dwóch poziomach, kluczowym aspektem jest odpowiedni podział zadań między tymi poziomami. Założono również, że oba poziomy powinny mieć możliwość przeprowadzania symulacji obiektu, aby można było weryfikować poprawność funkcjonowania obu w niezależny sposób.

Pierwszy z nich to część ,,wyższego'' poziomu - została tak określona ze względu na to, iż nie ma bezpośredniego wpływu na kontrolowany fizyczny obiekt, a zajmuje się tylko modelem matematycznym. Jej zadania są przedstawione na poniższej liście.

\begin{enumerate}
    \item Symulacja modelu nieliniowego w pętli otwartej, aby zainicjować algorytm optymalizacji dynamicznej.
    \item Wyznaczanie sterowania czasooptymalnego dla zadanych wartości początkowych i końcowych.
    \item Symulacja modelu nieliniowego w pętli zamkniętej, aby zweryfikować wyliczone sterowanie czasooptymalne.
    \item Linearyzacja modelu w punkcie pracy.
    \item Wyznaczanie sterowania optymalnego w sensie liniowo-kwadratowego wskaźnika jakości (dla modelu linearyzowanego w punkcie pracy).
\end{enumerate}

Dodatkowo przyjęto następujące założenia w związku z tymi zadaniami (podzielone ze względu na to, którego zagadnienia optymalizacji dotyczą):

\begin{enumerate}
    \item Założenia związane ze sterowaniem czasooptymalnym:
    \begin{enumerate}
        \item Sterowanie czasooptymalne jest postaci ,,bang-bang'' (wyjaśnienie w sekcji \ref{sub:toc-nonlnr}).
        \item W związku z tym wystarczy wyznaczyć czasy przełączeń między konkretnymi sterowaniami maksymalnym i minimalnym oraz to, które z nich ma być aplikowane jako pierwsze.
    \end{enumerate}
    \item Założenia związane ze sterowaniem liniowo-kwadratowym:
    \begin{enumerate}
        \item Punkt pracy (w którym jest dokonywana linearyzacja) jest stanem docelowym zagadnienia czasooptymalnego, jeśli ten jest punktem równowagi systemu (definicja dana w podrozdziale \ref{sec:model}).
        \item Jeśli stan docelowy zagadnienia czasooptymalnego nie jest stanem ustalonym rozważanego układu, stosuje się przybliżenie go do pewnego punktu równowagi i tam dokonuje się linearyzacji modelu matematycznego.
        \item Sterowanie liniowo-kwadratowe jest wyliczane jako odchyłka od sterowania ustalonego dla punktu równowagi.
    \end{enumerate}
\end{enumerate}

Na podstawie powyższych zadań oraz założeń sformułowano poniższą listę parametrów, które musi przyjmować część optymalizacyjna od użytkownika:

\begin{itemize}
    \item parametry statyczne modelu matematycznego:
    \begin{itemize}
        \item fizyczne rozmiary zbiorników (parametry $a$, $b$, $c$, $R$, $w$ i $h_{max}$),
        \item opory wypływu ze zbiorników (parametry $C_{1}$, $C_{2}$ oraz $C_{3}$),
        \item współczynniki wypływu ze zbiorników(parametry $\alpha_{1}$, $\alpha_{2}$ i $\alpha_{3}$);
    \end{itemize}
    \item wielkości związane z ograniczeniami w zagadnieniu czasooptymalnym:
    \begin{itemize}
        \item maksymalne sterowanie ($u_{max}$),
        \item wartości początkowe poziomów wody w zbiornikach (parametry $h_{10}$, $h_{20}$ oraz $h_{30}$),
        \item wartości końcowe poziomów wody w zbiornikach (parametry $h_{1f}$, $h_{2f}$ i $h_{3f}$);
    \end{itemize}
    \item wagi w zagadnieniu liniowo-kwadratowym:
    \begin{itemize}
        \item wartość wagi sterowania $R \in \mathbb{R}$,
        \item wartości wag stanów dane jako macierz $Q \in \mathbb{R}^{3}$.
    \end{itemize}
\end{itemize}

Drugi element zaproponowanego w niniejszej pracy oprogramowania to część ,,niższego'' poziomu - jej nazwa jest związana z tym, że bezpośrednio wpływa na sterowany układ. Nie zawiera tak skomplikowanych narzędzi obliczeniowych, ale za to powinna cechować się dużą niezawodnością. Jego zadania są przedstawione na poniższej liście.

\begin{enumerate}
    \item Aplikacja sterowania czasooptymalnego przez czas, który jest wyliczony przez część ,,wyższego poziomu'' oraz w odpowiedniej postaci (,,bang-bang'').
    \item Po upływie tego czasu aplikacja sterowania optymalnego w sensie liniowo-kwadratowego wskaźnika jakości aż do czasu otrzymania kolejnego sterowania czasooptymalnego.
    \item Symulacja modelu matematycznego układu w celach testowych i weryfikacyjnych.
\end{enumerate}

W związku z tym, że jest to element realizujący bezpośrednie sterowanie, użytkownik nie powinien mieć wpływu na jego funkcjonowanie. Cały interfejs między nim a programem sterującym powinien dotyczyć tylko i wyłącznie części ,,wyższego poziomu''.


\section{Komunikacja między elementami oprogramowania}
\label{sec:komunikacja}

Hybrydowa struktura aplikacji wymusza dokładne zdefiniowanie schematów komunikacyjnych między oboma jej poziomami.
Poniżej znajduje się podsumowanie wartości wysyłanych przez oba poziomy.

\begin{enumerate} 
    \item Część obliczeniowa wysyła:
    \begin{enumerate}
        \item wartości końcowe poziomów w zbiornikach (będące stanem docelowym w zadaniu czasooptymalnym oraz punktem linearyzacji modelu służącym do wyliczenia nastaw regulatora liniowo-kwadratowego),
        \item dla regulatora czasooptymalnego:
        \begin{enumerate}
            \item czasy przełączeń (założono postać sterowania typu ,,bang-bang''),
            \item wartość początkową sterowania czasooptymalnego,
            \item wartość ,,drugorzędną'' tego sterowania (założono, że część ,,niższa'' nie musi znać ograniczeń nałożonych na sterowanie),
            \item czas aplikacji sterowania czasooptymalnego (będący wartością wskaźnika jakości w tym zadaniu);
        \end{enumerate}
        \item dla regulatora liniowo-kwadratowego:
        \begin{enumerate}
            \item wektor K współczynników regulatora,
            \item sterowanie ustalone, od którego są liczone odchyłki.
        \end{enumerate}
    \end{enumerate}
    \item Część sterowania bezpośredniego wysyła:
    \begin{itemize}
        \item aktualne poziomy wody w zbiornikach,
        \item aktualną wartość sterowania.
    \end{itemize}
\end{enumerate}

Wyszczególniono 3 możliwe akcje zachodzące w systemie, poniżej znajduje się ich lista. Zilustrowano je odpowiednimi (aczkolwiek uproszczonymi do poziomu ogólnej specyfikacji) diagramami sekwencji według konwencji UML. Wyszczególniono na nich użytkownika, część obliczeniową i sterującą oraz oprogramowanie optymalizacyjne, aby zaznaczyć, że jest ono de facto osobnym elementem, zewnętrznym i niestanowiącym części aplikacji będącej przedmiotem niniejszej pracy.

\begin{enumerate}
    \item Akcja obliczania sterowania czasooptymalnego (przedstawiona na rys. \ref{fig:comm-toc}).
    \item Akcja obliczania sterowania liniowo-kwadratowego (pokazana na rys. \ref{fig:comm-lqr}).
    \item Akcja wysłania obliczonych sterowań między poziomami aplikacji (zaprezentowana na rys. \ref{fig:comm-send} zawierającym pozostałe akcje w uproszczonej formie).
\end{enumerate}

\begin{figure}[hpt]
    \centering
    \includegraphics[width=\textwidth]{Grafika/communication-toc}
    \caption{Diagram sekwencji ilustrujący akcję obliczenia sterowania czasooptymalnego. Źródło: własne.}\label{fig:comm-toc}
\end{figure}

\begin{figure}[hpt]
    \centering
    \includegraphics[width=\textwidth]{Grafika/communication-lqr}
    \caption{Diagram sekwencji ilustrujący akcję obliczenia sterowania liniowo-kwadratowego. Źródło: własne.}\label{fig:comm-lqr}
\end{figure}

\begin{figure}[hpt]
    \centering
    \includegraphics[width=\textwidth]{Grafika/communication-between-levels}
    \caption{Diagram sekwencji ilustrujący akcję wysyłania sterowań. Źródło: własne.}\label{fig:comm-send}
\end{figure}
