\section{Podział zadań między elementami oprogramowania}
\label{sec:podzial-zadan}

W związku z założoną architekturą oprogramowania, opierającą się na dwóch częściach, kluczowym aspektem jest odpowiedni podział zadań między nimi oraz wybór parametrów wysyłanych przez obie strony.

Pierwsza z nich to część ,,wyższego poziomu'' - została tak określona ze względu na to, iż nie ma bezpośredniego wpływu na kontrolowany fizyczny obiekt. Jej zadania są przedstawione poniżej.

\begin{enumerate}
    \item Zadania części ,,wyższego poziomu'':
    \begin{enumerate}
        \item wyznaczanie sterowania czasooptymalnego dla zadanych wartości początkowych i końcowych,
        \item symulacja modelu nieliniowego, aby zweryfikować wyliczone sterowanie czasooptymalne,
        \item linearyzacja modelu w punkcie pracy, czyli w stanie docelowym zagadnienia czasooptymalnego,
        \item wyznaczanie sterowania optymalnego w sensie liniowo - kwadratowego wskaźnika jakości (dla modelu linearyzowanego w punkcie pracy).
    \end{enumerate}
    \item Wysyła ona:
    \begin{enumerate}
        \item wartości końcowe poziomów w zbiornikach (będące stanem docelowym w zadaniu czasooptymalnym oraz punktem linearyzacji modelu służącym do wyliczenia nastaw regulatora liniowo-kwadratowego),
        \item dla regulatora czasooptymalnego:
        \begin{enumerate}
            \item czasy przełączeń (założono postać sterowania typu ,,bang-bang''),
            \item wartość początkową sterowania czasooptymalnego,
            \item wartość ,,drugorzędną'' tego sterowania (założono, że część ,,niższa'' nie musi znać ograniczeń nałożonych na sterowanie),
            \item czas aplikacji sterowania czasooptymalnego (będący wartością wskaźnika jakości w tym zadaniu),
        \end{enumerate}
        \item dla regulatora liniowo-kwadratowego:
        \begin{enumerate}
            \item wektor K współczynników regulatora,
            \item sterowanie ustalone, od którego są liczone odchyłki.
        \end{enumerate}
    \end{enumerate}
\end{enumerate}