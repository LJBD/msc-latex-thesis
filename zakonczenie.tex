\chapter*{Zakończenie}
\addcontentsline{toc}{chapter}{Zakończenie}
\label{cha:zakonczenie}

Przygotowanie aplikacji realizującej postawione we wstępie niniejszej pracy cele okazało się zadaniem tyleż trudnym, co fascynującym. Obie te cechy wynikały przede wszystkim z założeń postawionych przed wyższym poziomem aplikacji opisanych na początku podrozdziału \ref{sec:czesc-wyzsza}. Znalezienie i opisanie otwartego oprogramowania, które potrafi rozwiązywać problemy optymalizacji dynamicznej układów nieliniowych i posiada przystępny interfejs użytkownika, jest niewątpliwą wartością tej pracy.

Udało się zrealizować wszystkie cele: pokazano, iż wyższy poziom aplikacji jest w stanie wyznaczać sterowanie optymalne w rozumieniu dwóch wskaźników jakości: liniowo-kwadratowego i czasowego. Umożliwia on również przeprowadzenie symulacji weryfikacyjnych wyznaczonych sterowań, które opisano w rozdziale \ref{cha:symulacja}. Niższy poziom aplikacji został przygotowany z myślą o komunikacji z układem pomiarowo-sterującym, ale ona nie została przetestowana. Używano tego poziomu tylko w celach symulacyjnych, co dało możliwość sprawdzenia funkcjonowania wyznaczonego sterowania optymalnego poza środowiskiem, w którym zostało obliczone.

Podstawowym kierunkiem dalszych prac byłoby połączenie napisanej aplikacji z rzeczywistym obiektem, co na pewno dostarczyłyby istotnych wniosków co do funkcjonowania całości aplikacji. Byłoby dobrze uzupełnić również opis problemu o identyfikację elementów wykonawczych oraz wyznaczenie zbioru stanów osiągalnych, aby ograniczyć użytkownikowi zbiór możliwych punktów końcowych optymalizacji. Można by również przygotować układ predykcyjny, w celu poprawy błędów sterowania obliczanego w czasie działania aplikacji. Zapewne byłoby to potrzebne przy użyciu jej z rzeczywistym układem zbiorników.

Poza tym należałoby udoskonalić inicjalizację optymalizacji, gdyż to zagadnienie nie zostało rozpoznane dogłębnie w niniejszej pracy. Jest to jednak problem na tyle złożony, że z powodzeniem mógłby stać się tematem kolejnej pracy dyplomowej. Nie testowano również wszystkich możliwości dostosowania algorytmu optymalizacji do potrzeb tego zagadnienia.
