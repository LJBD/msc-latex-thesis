\documentclass[12p]{article}

\usepackage[polish]{babel}
\usepackage{polski}
\usepackage[utf8]{inputenc}

%\DeclareMathSizes{10}{10}{10}{8}

\begin{document}
	\begin{equation}\label{eq:model}
		\left \{
		\begin{array}{lr}
			\frac{\partial h_{1}}{\partial t} = \frac{u - C_{1}\sqrt{h_{1}}}{aw} \\[8pt]
			\frac{\partial h_{2}}{\partial t} = \frac{C_{1}\sqrt{h_{1}} -  C_{2}\sqrt{h_{2}}}{cw + \frac{h_{2}}{h_{max}}bw} \\[20pt]
			\frac{\partial h_{3}}{\partial t} = \frac{C_{2}\sqrt{h_{2}} -  C_{3}\sqrt{h_{3}}}{w\sqrt{R^{2} - (R - h_{3})^{2}}}
		\end{array}
		\right.
	\end{equation}
	
	Model matematyczny zestawu zbiorników (\ref{eq:model}), gdzie:
	\begin{itemize}
		\item $a$ - szerokość pierwszego zbiornika,
		\item $b$ - szerokość trójkątnej części drugiego zbiornika,
		\item $c$ - szerokość prostopadłościennej części drugiego zbiornika,
		\item $R$ - promień trzeciego zbiornika,
		\item $w$ - głębokość zbiorników,
		\item $h_{max}$ maksymalna wysokość słupa wody w zbiornikach,
		\item $C_{1}$ - współczynnik wypływu z pierwszego zbiornika,
		\item $C_{2}$ - współczynnik wypływu z drugiego zbiornika,
		\item $C_{3}$ - współczynnik wypływu z trzeciego zbiornika.
	\end{itemize}

	\begin{equation}\label{eq:sprzezone}
		\left \{
		\begin{array}{lr}
			\frac{\partial \psi_{1}}{\partial t} =  \psi_{1}\frac{C_{1}}{2aw\sqrt{h_{1}}} - \psi_{2}\frac{C_{1}}{2\sqrt{h_{1}}(cw + \frac{h_{2}}{h_{max}}bw)} \\[20pt]
			\frac{\partial \psi_{2}}{\partial t} = - \psi_{2}\frac{1}{cw + \frac{h_{2}}{h_{max}}bw}(\frac{b(C_{1}\sqrt{h_{1}} - C_{2}\sqrt{h_{2}})}{ch_{max} + bh_{2}} - \frac{C_{2}}{2\sqrt{h_{2}}}) - \psi_{3}\frac{1}{w\sqrt{h_{3}(2R - h_{3})}} \\[20pt]
			\frac{\partial \psi_{3}}{\partial t} = \psi_{3}\frac{-C_{3}(3R - 2h_{3})}{wh_{3}(2R - h_{3})^{\frac{3}{2}}}
		\end{array}
		\right.
	\end{equation}
	
	Równania sprzężone zestawu zbiorników (\ref{eq:sprzezone}).

\end{document}